
%
% This file was prepared using 
% AMS-LaTeX2e <1996/06/01>
% Document Class: amsart 1995/02/23 v1.2b
% Authors:
% Title: 
% Date:
%

\documentclass[12pt]{amsart}
%\usepackage{amssymb}
\setlength{\textwidth}{17.5cm}\oddsidemargin=-1cm\evensidemargin=-1cm
\setlength{\textheight}{20cm}
\begin{document}
\numberwithin{equation}{section}

\def\1#1{\overline{#1}}
\def\2#1{\widetilde{#1}}
\def\3#1{\widehat{#1}}
\def\4#1{\mathbb{#1}}
\def\5#1{\frak{#1}}
\def\6#1{{\mathcal{#1}}}

\def\C{{\4C}}
\def\R{{\4R}}
\def\N{{\4N}}
\def\Z{{\4Z}}

\title[]{Approximate geometry and Catlin's property P}
\author[D. Zaitsev]{Dmitri Zaitsev}
\address{D. Zaitsev: School of Mathematics, Trinity College Dublin, Dublin 2, Ireland}
\email{zaitsev@maths.tcd.ie}
%\subjclass{}

\maketitle
\tableofcontents


\def\Label#1{\label{#1}{\bf (#1)}~}
%\def\Label#1{\label{#1}}



% Standard sets

\def\cn{{\C^n}}
\def\cnn{{\C^{n'}}}
\def\ocn{\2{\C^n}}
\def\ocnn{\2{\C^{n'}}}

% Abbreviations

\def\dist{{\rm dist}}
\def\const{{\rm const}}
\def\rk{{\rm rank\,}}
\def\id{{\sf id}}
\def\aut{{\sf aut}}
\def\Aut{{\sf Aut}}
\def\CR{{\rm CR}}
\def\GL{{\sf GL}}
\def\Re{{\sf Re}\,}
\def\Im{{\sf Im}\,}
\def\span{\text{\rm span}}


\def\codim{{\rm codim}}
\def\crd{\dim_{{\rm CR}}}
\def\crc{{\rm codim_{CR}}}

\def\phi{\varphi}
\def\e{\varepsilon}
\def\eps{\varepsilon}
\def\d{\partial}
\def\a{\alpha}
\def\b{\beta}
\def\g{\gamma}
\def\G{\Gamma}
\def\D{\Delta}
\def\Om{\Omega}
\def\k{\kappa}
\def\l{\lambda}
\def\L{\Lambda}
\def\z{{\bar z}}
\def\w{{\bar w}}
\def\Z{{\1Z}}
\def\t{\tau}
\def\th{\theta}
\def\p{\phi}
\def\de{\delta}
\def\la{\langle}
\def\ra{\rangle}
\def\r{\rho}

\emergencystretch15pt
\frenchspacing

\newtheorem{Thm}{Theorem}[section]
\newtheorem{Cor}[Thm]{Corollary}
\newtheorem{Pro}[Thm]{Proposition}
\newtheorem{Lem}[Thm]{Lemma}

\theoremstyle{definition}\newtheorem{Def}[Thm]{Definition}

\theoremstyle{remark}
\newtheorem{Rem}[Thm]{Remark}
\newtheorem{Exa}[Thm]{Example}
\newtheorem{Exs}[Thm]{Examples}

\def\bl{\begin{Lem}}
\def\el{\end{Lem}}
\def\bp{\begin{Pro}}
\def\ep{\end{Pro}}
\def\bt{\begin{Thm}}
\def\et{\end{Thm}}
\def\bc{\begin{Cor}}
\def\ec{\end{Cor}}
\def\bd{\begin{Def}}
\def\ed{\end{Def}}
\def\br{\begin{Rem}}
\def\er{\end{Rem}}
\def\be{\begin{Exa}}
\def\ee{\end{Exa}}
\def\bpf{\begin{proof}}
\def\epf{\end{proof}}
\def\ben{\begin{enumerate}}
\def\een{\end{enumerate}}
\def\beq{\begin{equation}}
\def\eeq{\end{equation}}


\section{Introduction}
\subsection{Overview and motivation}
This is the first paper in the series 
aiming at understanding geometric invariants
behind the tools developed by Catlin
in his celebrated papers 
\cite{C84a, C84b, C87}
in order to obtain a priori estimates for the $\bar\d$ operator,
motivated by previous foundational work of Kohn \cite{K64a, K64b, K79}.
%on regularity properties for solutions of $\bar\d$ equations.
%It is believed that proper understanding
%of Catlin's machinery may lead
%to new constructions of a priori estimates 
%also for other classes of
%Partial Differential Operators
%under natural geometric conditions.
%In view of somewhat challenging complexity of Catlin's original proofs,
%there is considerable interest in the community 
%in having a more accessible and simplified treatment.

A geometric aspect of Catlin's a priori estimates proof
consists of showing the existence of {\em weight functions}
% (called weight functions by Catlin \cite{C87})
 satisfying certain boundedness and positivity estimates for their complex Hessians,
 that are known as ``Property (P)'' type conditions 
 (see e.g.\ \cite{BS16} for a recent survey).
A major difficulty when constructing such weight functions
under geometric conditions (such as D'Angelo's finite type \cite{D82}), 
is to keep the uniform 
nature of the estimates across points 
of {\em varying ``degeneracy''}
for the underlying geometry.
A simple example of a degeneracy measure
is the {\em rank of the Levi form} of the 
boundary $M:=\d D$ (where $D$ is a domain in $\cn$).
A more refined measure
is Catlin's multitype \cite{C84b}, see also \S\ref{m-type}.
To deal with points of varying multitype, 
Catlin developed his machinery of boundary systems \cite{C84b}
to gain control of the multitype level sets
by including them locally into certain {\em ``containing submanifolds''}
transversal to the Levi form kernel.
A result of this type is the content of 
\cite[Main Theorem, Part (2)]{C84b},
where a containing submanifold
is constructed by a collection of 
inductively chosen {\em boundary system functions}
that arise as certain carefully selected
(vector field) derivatives of the Levi form.

In this paper we focus on geometric invariants
behind the containing submanifold construction,
which in some way extend and simplify the boundary system approach.
To make things more concrete and explicit,
we restrict ourselves to the
$4$th order invariants that, in particular,
cover the cases of pseudoconvex hypersurfaces of the finite type
at most $4$,
%the minimal finite type occurring
%for Levi-degenerate pseudoconvex hypersurfaces.
where the mutitype level sets
boil down to simpler {\em level sets of the
Levi (form) rank} (see Proposition~\ref{multi-quartic} for details).
%Another motivation is that even for the finite type $4$,
%the subelliptic exponent $\epsilon$ given in \cite{C87} 
%(page 133)
%seems to be fairly rough and far from optimal.
Recall that Catlin's boundary system functions
\cite{C84b}
are constructed inductively with every new equation
depending on chosen solutions for previous ones.
In comparison, we here collect natural defining functions 
for the Levi rank level sets
into {\em invariant ideal sheaves $\6I(q)$} on $M$,
for each Levi rank $q$.
The sheaf $\6I(q)$ is generated
by certain $1$st order Levi form derivatives as
described in Theorem~\ref{main} below.
In particular, arbitrary defining functions from $\6I(q)$
can be combined without any additional relations.
Furthermore, additional derivatives
of the Levi form along arbitrary complex vector fields $L^3$
(in Theorem~\ref{main}, Part (5)),
including transversal ones,
 are allowed for functions in $\6I(q)$.
In comparison, for a related boundary system function
given by the same formula,
the outside vector field $L^3$
would have to be in a special subbundle
inside the holomorphic tangent bundle.
As a result, we obtain richer classes of defining functions
allowing for more control over containing submanifolds
(see Example~\ref{transv-f}),
that may potentially lead to sharper a priori estimates.

%In Catlin's work \cite{C84b}, such degree of degeneracy
%of
%is measured by an $n$-tuple 
%of rational numbers
%$$
%	\6M( p)=(m_1, \ldots, m_n) \in \4Q^n,
%$$
%called the {\em multitype} of $M$ at $p$
%
%We are going to focus here on constructing invariants

%To put things in better perspective,
%recall briefly that a significant step in 
%the subelliptic estimates proof \cite{C87}
%is based on constructing families of bounded potential functions
%in strips around the given hypersurface, 
%with their hermitian Hessians growing 
%at least as a certain power of the strip's width.
%The construction of the potential functions
%proceeds by induction on certain (approximate) multitypes
%associated to so-called {\em approximate boundary systems}.
%The induction step relies 
%on isolating sets of constant multitypes
%by including them into certain submanifolds and 
%pushing the potential estimate away
%from each level set towards
%points of lower multitype.
%The required multitype and boundary system foundations
%without the approximate contexts
%were previously laid out by Catlin in his earlier paper \cite{C84},
%where one of the main results -- Main Theorem, Part (2),
%is precisely about constructing submanifolds
%containing multitype level sets.
%


%We are going to focus on the geometric part of 


%One of our goals here is to lay some foundations 
%to be used in forthcoming papers aiming
%at giving such a treatment.
%A parallel goal is to make the construction more invariant
%by defining certain canonical tensors
%and sheaves of vector fields and functions
%invariantly associated with the underlying CR geometry.
%

In parallel to the ideal sheaf $\6I(q)$ construction,
we introduce {\em invariant quartic tensors} $\t^4$,
giving a precise control over differentials of
the functions in $\6I(q)$.
This is expressed in Theorem~\ref{main}, part (2),
where the tangent space of the containing manifold $S$
equals the real kernel of the tensor.
Importantly, the full tangent space of $S$
(rather than only the tangential part) is controlled here
via the kernel of $\t^4$, which means that transversal vector fields
must also be allowed among tensor arguments.
The tensors are constructed in Lemma~\ref{d2}
as certain $2$nd order Levi form derivatives
taken along all possible vector fields.
In comparison, only derivatives with 
respect to $(1,0)$ and $(0,1)$ vector fields
can appear in the boundary systems.
%
%to prove his celebrated theorem
%on {\em subelliptic estimates \cite{C87}
%for pseudoconvex 
%real hypersurfaces in $\C^n$ of finite type}.
%
%Subelliptic estimates are certain a priori estimates
%for the $\bar\d$ operator
%that appeared in the work of Kohn \cite{K64}
%and play important role in establishing 
%smooth boundary regularity for solutions 
%of $\bar\d$ equation.
%and have the form
%$$
%	|||\phi|||^2_{\e} \le C (
%		\|\bar\d\phi\|^2
%		+
%		\|\bar\d^*\phi\|^2
%		+
%		\| \phi \|^2
%	),
%$$
%where $\phi$ is a $(p,q)$-form
%
% lies in the fact
%(in the sense of D'Angelo \cite{D82}).
%

For reader's convenience,
we summarize the main results and constructions
in Theorem~\ref{main},
leaving more detailed and general statements
with their proofs in the chapters following.


%It is noteworthy to mention the importance
%of combining both coordinate based multitype and 
%coordinate-free
%vector field based boundary system approaches
%in Catlin's theory, each with its own advantages and limitations.
%The same dichotomy also becomes crucial in our approach.
%The coordinate-free vector field based construction
%gives immediate invariance for the tensors, 
%whereas their properties such as symmetries
%are often easier verified via calculations in 
%the normal form coordinates.

\subsection{More details on invariant tensors and sheaves}
Our first step in defining invariant tensors is a byproduct result 
giving a {\em complete set of cubic invariants}
for a general real hypersurface $M$, without pseudoconvexity assumption.
This is achieved by constructing an invariant cubic tensor $\t^3$
obtained by differentiating the Levi form along vector fields
with values in the Levi kernel, see Lemma~\ref{levi-der}.
Remarkably, to obtain tensoriality of the Levi form derivatives, 
it is of crucial importance to require
{\em both vector fields inside the Levi form to take values in the Levi kernel}
as explained in Example~\ref{one-ker}.
This stands further, in remarkable contrast with the cubic tensor $\psi_3$
%on the compexified tangent bundle $\C T$ of $M$.
% at a point $p$:
%$$
%	\t^3_p \colon \C T_p \times \C T_p \times \C T_p \to \C T_p/\C H_p,
%$$
%where $\C H_p$ is the complexified holomorphic tangent space.
%A related cubic tensor has been constructed 
defined by Ebenfelt \cite{E-jdg}
(by means of the Lie derivatives of the contact form),
where only one of the arguments needs to be in the Levi kernel.
%
%with remaining part obtained here
%by differentiating the Levi form in the transversal direction
%to the complex tangent.
On the other hand, Ebenfelt's tensor $\psi_3$
does not allow for transversal directions as $\t^3$ does.
It turns out that the pair $(\psi_3, \t^3)$ yields
a complete set of cubic invariants, 
as demonstated by a normal form 
(of order $3$)
in Proposition~\ref{3-normal} eliminating all other terms
that are not part of either of the tensors.

We also investigate the construction based on double Lie brackets
(also considered by Webster \cite{W}).
This approach, however, forces all vector fields
to be in the complexified holomorphic tangent bundle,
leading only to a restriction of the tensor $\t^3$.
Again, the double Lie bracket construction
is only tensorial when both vector fields inside the inner bracket
take their values in the Levi kernel (see Example~\ref{one-ker}).

As mentioned earlier, the cubic tensor $\t^3$ is constructed without any pseudoconvexity assumption.
On the other hand, in presence of pseudoconvexity,
the whole tensor $\t^3$
must vanish identically (Lemma~\ref{psc-vanish}).
The only cubic terms that may survive are of the form \eqref{psc-cubic}
which can never appear in the lowest weight terms,
and hence never play a role in Catlin's multitype 
and boundary system theory.  
%Consequently, cubic terms do not suffice 
%to capture the missing Levi nondegeneracy
%at the lowest weight level.

Motivated by the above, we next look for quartic tensors.
It turns out (Example~\ref{ex-4}) 
that this time, neither second order Levi form derivatives
nor quartic Lie brackets provide tensorial invariants
even when all vector field arguments take their values in the Levi kernel.
To overcome this problem, we {\em restrict the choice of 
the vector fields involved} by requiring a certain kind of condition of
 ``Levi kernel inclusion up to higher order'' (Definition~\ref{ker-1}).
% In a nutshell, this condition requires the vector 
 In Lemma~\ref{1-ker-def} we show that
this additional condition always holds
for any vector field that is Levi-orthogonal
to a maximal Levi-nondegenerate subbundle,
which, in particular, are some of the vector fields 
in a Catlin's boundary system.
However, the mentioned Levi-orthogonality lacks some invariance
as it depends on the choice of the subbundle.
In contrast, the Levi kernel inclusion up to order $1$
is invariant and only depends
on the $1$-jet of the vector field at the reference point.


With that restriction in place, an invariant quartic tensor $\t^4$
can now be defined in a similar fashion.
%We also obtain an improved normal form
Then
it's restriction $\t^{40}$ enters 
the lowest weight normal form 
with weights $\ge 1/4$,
see Proposition~\ref{4-normal}.
It turns out, the restriction $\t^{40}$ 
provides exactly the missing information 
at the lowest weight level for hypersurfaces of finite type $4$
(where finite type $3$ cannot occur for pseudoconvex points in view of
Corollary~\ref{pse-nf-cor}).
For example, both D'Angelo finite type $4$
and Catlin's multitype up to entry $4$
can be completely characterized in terms
of $\t^{40}$.
In fact, having the finite type $4$ is equivalent
to the nonvanishing of $\t^{40}$ on complex lines
(Proposition~\ref{type-quartic}),
whereas having a multitype up to entry $4$
is equivalent to $\t^{40}$ having trivial kernel
(Propositions~\ref{multi-quartic}).

In \S\ref{ideal} we use the quartic tensor $\t^4$ to 
characterize the differentials of functions in 
the ideal sheaf $\6I(q)$
%(by the very definition of $\t^3$ via Lemma~\ref{d2})
as well as the minimal tangent spaces
of containing manifolds
defined by a transversal set of functions in $\6I(q)$.

Finally, in \S\ref{bs} we obtain a characterization 
for a Catlin's boundary system,
where the most difficult part 
of obtaining vector field directions of nonvanishing Levi form derivatives
at the lowest weight 
is replaced by the nonvanishing of the tensor $\t^{40}$
on the vector fields' values at the reference point,
a purely algebraic condition.



In a forthcoming paper will shall 
extend the present 
geometric approach 
towards its {\em approximate versions}
with necessary control
to perform the induction step
in the subelliptic estimate proof.

\bigskip

{\bf Acknowledgements.}
The author would like to thank
J. J. Kohn, D. W. Catlin, J. P. D'Angelo,
E. J. Straube, M. Kolar, S. Fu,
J. D. McNeal and A. C. Nicoara
for numerous inspiring discussions.




\section{Notation and main results}
We shall work in the smooth ($C^\infty$) category
unless stated otherwise.
Let $M\subset \C^{n}$ be a real hypersurface.
We write $T:= TM$ for its tangent bundle, 
$H = HM \subset  T$ for the complex (or holomorphic) tangent bundle,
$Q:= T/H$ for the normal bundle,
as well as 
$$
	\C T := \C\otimes T,
	\quad 
	\C H: =\C\otimes H,
	\quad
	\C Q:= \C\otimes Q,
$$
for their respective complexifications.
Further, $H^{10}$ and $H^{01}= \1{H^{10}}$ denote $(1,0)$ and $(0,1)$ bundles respectively, such that $\C H = H^{10}\oplus \1{H^{10}}$.
By a slight abuse of notation, we write $L\in V$
when a vector field $L$ is a local section in $V$,
which can be a bundle or a sheaf.

On the dual side, 
$\Om$ stands for the bundle of all real $1$-forms, 
$\C \Om$ for all complex $1$-forms,
$\Om^0$ for all {\em contact forms}, 
i.e.\ forms form $\Om$ that are vanishing on $H$ and real-valued on $T$,
and $\C\Om^0$ for the corresponding complexification.

Recall that a {\em (local) defining function} of $M$
is any real-valued function $\rho$ with $d\rho\ne 0$ such that $M$ given by 
$\rho=0$. For any defining function $\rho$, the one-form 
$\th: = i\d\rho$ 
spans (over $\R$) the bundle $\Om^0$ of all contact forms. 
 
We shall consider the standard pairing $\la \th, L\ra := \th(L)$
for $\th\in \C\Om$, $L\in \C T$.
By a slight abuse, we keep the same notation also for the pairing
$$
	\la \cdot, \cdot \ra \colon \C\Om^0 \times \C Q \to \C
$$
between the (complex) contact forms and the normal bundle.
With this notation, we regard the {\em Levi form} at a point $p\in M$ 
as  $\C$-bilinear map
$$
	\t^2_p \colon H^{10}_p \times \1{H^{10}_p} \to \C Q_p,
$$
which
is uniquely determined by the identity
\beq\Label{levi-id}
	\la \th_p, \t^2_p(L^2_p, L^1_p) \ra = - i \la \th, [L^2, L^1] \ra_p,
	\quad 
	L^2\in H^{10}, \, L^1\in \1{H^{10}}.
\eeq
The normalization of $\t^2$ used here is chosen
such that for the quadric
$$
	\rho = -2\Re w + q(z,\z) =0,
	\quad
	(w,z)\in \C\times \C^{n-1},
$$ 
%the vector fields
%$$
%	L_j := \d_{z_j} + \z_j \d_w
%$$
and the contact form $\th = i\d \rho$,
we have
$$
	\la \th_0,  \t^2_0(\d_{z_j}, \d_{\z_k}) \ra = \d_{z_j} \d_{\z_k} q,
$$
or more generally
\beq\Label{levi-calc}
	\la \th_0,  \t^2_0(v^2, v^1) \ra = \d_{v^2} \d_{v^1} q,
\eeq
where $v^2 \in H^{10}_0$, $v^1\in \1{H^{10}_0}$.
Here $\d_v$ denotes the directional derivative 
along the vector $v$, which is thought to be applied at every point.
%We further use \eqref{levi-calc} to extend $\l_0$
%to a tensor $\C H_0 \times \C H_0\to \C Q_0$.

%Motivated by this calculation, we extend the Levi form $\l$
%canonically to a (unique) symmetric tensor
%\beq\Label{levi-ext}
%	\t^2_p \colon \C H_p\times \C H_p \to \C Q_p,
%\eeq
%which vanishes on 
%$H^{10}_p\times H^{10}_p$ 
%and
%$H^{01}_p\times H^{01}_p$.
%That is, $\t^2$ extends the Levi form by symmetry and by vanishing
%in the remaining arguments.
%%Equivalently 
%In addition, $\t^2_p$ satisfies the reality condition
%$$
%	\1{\t^2_p(v^2, v^1)} = \t^2_p(\1v^2, \1v^1).
%$$
%Furthermore, for the qua

We say that a {\em point $p\in M$ is of Levi rank $q$},
if the Levi form $\t^2_p$ at $p$ has rank $q$.
A subbundle $V\subset H^{10}$
is called {\em Levi-nondegenerate},
if the Levi form is nondegenerte on $V\times \1V$. 
For every such subbundle $V$,
we write 
%$$
%	V^\perp := \{ v\in H^{10} : \t^2(v, v^1) =0   \} 
%$$
$$
	V^\perp \subset H^{10},
	\quad
	V^\perp = \cup_x V_x^\perp,
	\quad
	V_x^\perp = \{ v\in H^{10}_x : \t^2_x(v, \1v^1) =0 \text{ for all } v^1\in V\},
$$
for the orthogonal complement with respect to the Levi form,
which is necessarily a subbundle.

Finally, we write $K^{10}_p\subset H^{10}_p$
and $K^{01}_p = \1{K^{10}_p}\subset H^{01}_p$
for the Levi kernel components at $p$, 
$\C K_p= K^{10}_p \otimes \1{K^{10}_p}$
for the complexification and
$K_p = \C K_p \cap T_p$
for the corresponding real part.


%\subsection{Weight notation}
\bigskip
The following is an overwiev of some of the main results:


\bt\Label{main}
Let $M\subset\cn$ be a pseudoconvex real hypersurface.
Then for every $q\in \{0, \ldots, n-1\}$, there exist 
an invariant submodule sheaf $\6S^{10}(q)$ of $(1,0)$ vector fields,
an invariant ideal sheaf $\6I(q)$ of complex functions,
and for every $p\in M$ of Levi rank $q$,
an invariant quartic tensor
$$
	\t^4_p \colon  \C T_p \times \C T_p 
	\times K^{10}_p\times \1{K^{10}_p} \to \C Q_p,
$$
and
a real submanifold $S\subset M$ through $p$,
such that the following hold:

\begin{enumerate}

\item 
$S$ contains the set of all points $x\in M$ of Levi rank $q$
in a neighborhood of $p$.

\item
The tangent space of $S$ at $p$ equals the real part of the kernel of $\t^4_p$:
$$
	T_p S = \Re\ker \t^4_p = \{ v\in T_p :  \t^4_p(v, v^3, v^2, v^1) = 0 
	\text{ for all } v^3, v^2, v^1 \}.
$$


\item
In suitable holomorphic coordinates vanishing at $p$,
$M$ admits the normal form
$$
	2\Re w = \sum_{j=1}^q |z_{2j}|^2 + \phi_4(z_4,\z_4) + o_w(1),
	\quad
	(w, z_2, z_4)\in \C\times \C^{q}\times \C^{n-q-1},
$$
where $o_w(1)$ has weight greater than $1$,
with weights $1$, $1/2$ and $1/4$ being assigned to
the components of $w$, $z_2$ and $z_4$ respectively,
and 
where $\phi_4$ is a homogenous polynomial of degree $4$
representing a restriction of the quartic tensor $\t^4_p$ in the sense that
$$
	\t^4_p(v^4, v^3, v^2, v^1) 
	= \d_{v^4} \d_{v^3} \d_{v^2} \d_{v^1} \phi_4
$$
holds for $v^4, v^3\in \C K_0$
and
$v^2, \1v^1\in K^{10}_0$.

\item
$S$ is given by 
$$
	S = \{f^1 = \ldots = f^m =0\},
	\quad
	df^1\wedge \ldots \wedge df^m \ne 0,
	\quad
	f^j \in \Re \6I(q).
$$
In fact, any $f\in \Re \6I(q)$ vanishes
on the set of point of Levi rank $q$. 

\item
The ideal sheaf $\6I(q)$
is generated by all functions $f$ of the form
$$
	f = L^3 \la \th, [L^2, L^1] \ra,
	\quad 
$$
where $\th\in\Om^0$ is a contact form, 
$L^3\in \C T$ arbitrary complex vector field,
and $L^2, \1L^1 \in \6S^{10}(q)$
arbitrary sections of the submodule sheaf.

\item
The submodule sheaf $\6S^{10}(q)$
is generated by all $(1,0)$ vector fields $L$
satisfying $L\in V_L^\perp$,
with $V_L\subset H^{10}$
a Levi-nondegenerate subbundle
of rank $q$ in a neighborhood of $p$.

\end{enumerate}


In addition, when $M$ is of finite type at most $4$ at $p$,
the following also holds:


\begin{enumerate}

\item[(i)]
The intersection $T_p S\cap K_p$ with the Levi kernel $K_p$ is totally real.
\item[(ii)]
For every $v\in K^{10}_p$, the tensor $\t^4_p$ does not identically 
vanish on
$$
	(\C v + \C \1v) \times (\C v + \C \1v) \times \C v\times \C \1v.
$$
\item[(iii)]
The (Catlin's) multitype at $p$ equals
%\beq\Label{multitype}
$$
	(1,2,\ldots, 2, 4,\ldots, 4),
$$
%\eeq
where the number of $2$'s equals the Levi rank at $p$.
In particular, the multitype is determined by the Levi rank.

\end{enumerate}
\et

For proofs and more detailed and general statements,
see the respective sections below.
Part (6) can be used to define 
the submodule sheaves $\6S^{10}(q)$,
see \S\ref{submodules}.
Then the ideal sheaves $\6I(q)$ are defined in Part (5),
see also \S\ref{ideal}.
In particular, local sections in $\6I(q)$
vanish at points of Levi rank $q$
by Corollary~\ref{iq-van}.
The quartic tensor $\t^4$ is constructed 
\S\ref{quartic}.
In view of Proposition~\ref{main0},
the intersection of real kernels of differentials
$df$ for $f\in \Re \6I(q)$ coincides
with $\Re \ker \t^4_p$.
Hence we can choose functions $f^j$
satisfying (4) and (2).
The normal form in (3) 
follows from Proposition~\ref{4-normal}.

When $M$ is of finite type $4$,
Proposition~\ref{type-quartic}
implies that $\t^4$
has no holomorphic kernel,
and therefore its (real) kernel as in (2) is totally real,
as stated in (i).
Statement (ii) is also part of Proposition~\ref{type-quartic}.
Finally, statement (iii) about the multitype 
is contained in \S\ref{m-type}.

\bigskip

The following simple example
illustrates one of the differences between
functions in the ideal sheaf $\6I(q)$
and  boundary systems
(as defined in \cite[\S2]{C84b},
see also \S\ref{bs} below).

\be\Label{transv-f}
Consider the hypersurface
$M\subset\C^2_{w,z}$ given by
$$
	2\Re w = \phi(z,\z, \Im w), \quad 
	\phi(z,\z,u):= |z|^4 + u^2|z|^2,
$$
which is pseudoconvex and of finite type $4$.
Then a boundary system $\{ L_2; r_2\}$
%
%can be chosen as
%$$
%	L_2 \equiv \d_z \mod \d_w,
%	\quad 
%	r_2 = \Re L_2 \d\phi([L_2, \1L_2])
%$$
defines the  $2$-dimensional
submanifold $S:=\{ r_2 =0\} \subset M$
to contain all points of Levi rank $0$.
However, since $r_2$ is of the form
\beq\Label{r2}
	r_2 = \Re L^3 \la \th, [L^2, L^1]\ra
\eeq
(in the setup of Theorem~\ref{main}, part (5)),
its differential at $0$ is given by
$$
	d r_2 (v) = \Re \t^4_0 (v, L^3_0, L^2_0, L^1_0),
$$
which vanishes on the transversal space $\{dz=0\}$.
Consequently, any $S$ defined by a boundary system function $r_2$
must be tangent to $\{dz=0\}$.
% defined by 
%$\Re z = 0$.
%Another boundary system would give $\Im z =0$.

On the other hand, 
in the ideal sheaf $\6I(0)$
we can choose a function given by \eqref{r2}
with {\em transversal $L^3$}.
That will allow to reduce $S$ down to only the origin 
$z=w=0$,
which, in fact, is the set of points of Levi rank $0$.
\ee



\section{Approximate Levi rank}
\subsection{$\de$-rank of an hermitian form}
Let $(V, h)$ be a complex vector space $V$
with a fixed hermitian metric $h$.
We write $\|v\|$ for the associated norm of $v\in V$.
Consider further a hermitian form
$\t\colon V\times \1V\to \C$  
and any $\de\ge 0$.
For every $v\in V$, consider the associated functional
$$
	\t_v=\t(v, \cdot) : \1V \to \C.
$$
Then $\t_v=0$ if and only if $v$ is in the kernel $\ker\t$.
We shall need the following notions quantifying
kernels and subspaces transversal to them:

\bd
%	The $\de$-kernel $\ker(\t;\de)$ 
	A {\em sub-$\de$-space (resp.\ super-$\de$-space)}
	is any vector subspace $W\subset V$ satisfying
	\beq\Label{tv}
		\|\t_v\| \le \de \|v\| 
		\quad
		(\text{resp. } \|\t_v\| > \de \|v\|),
		\quad
		v\in V.
	\eeq
	The {\em $\de$-rank of $\t$} is the maximal dimension of a
	super-$\de$-space.
	
	More generally, we call the {\em $\de$-kernel of $\t$}
	and denote by $\ker(\t; \de)$
	the set (cone) of all $v$ satisfying the first inequality in \eqref{tv}.
\ed

In other words, $W$ is a sub-$\de$-space if and only if
for all $v\in V$, $\w\in \1W$,
$$
	|\t(v,\w)| \le \de \|v\| \|\w\|,
$$
and is a super-$\de$-space if and only if
for every $v\in V$, there exists $\1w\in W$ with
$$
	 |\t(v,\w)| > \de \|v\| \|\w\|.
$$


\br
In the special case $\de=0$,
a sub-$0$-space is any subspace of $\ker\t$,
a super-$0$-space is any subspace transversal to $\ker\t$,
%where the map $v\mapsto \t_v$, $V\mapsto \1V^*$ is injective,
and the $0$-rank is the usual rank of $\t$.
\er

We have the following straightforward properties:

\bl
Let $\l_j\in \R$ be the eigenvalues of $\t$
with respect to $h$.
Then the $\de$-rank of $\t$
equals the number $q$ of the eigenvalues $\l_j$
satisfying $|\l_j| > \de$.
In particular, the $\de$-rank of $\t$ equals 
the maximal dimension of a vector subspace $W\subset V$ satisfying
$$
	|\t(v, \w)| > \de \|v\| \|\w\|, \quad v, w\in W.
$$
\el




\subsection{Application to the Levi form}
Similarly to the (exact) 
Levi kernel $K^{10}_p$
we write
$$
	K^{10}_p(\de) := \ker(\t^2_p; \de)  \subset H^{10}_p
$$
and call it the {\em Levi $\de$-kernel}.
Note that $K^{10}_p = K^{10}_p(0)$
but for $\de>0$, $K^{10}_p(\de)$ is not a subspace in general, only a cone.

\bd
We say that $p\in M$ is of {\em Levi $\de$-rank $q$}
whenever the Levi form at $p$ has $\de$-rank $q$.
\ed



\section{Invariant cubic tensors}

We begin by investigating the $3$rd order invariants
without the pseudoconvexity assumption.

\subsection{Double Lie brackets}
In presence of a nontrivial Levi kernel $K^{10}_p$,
it is natural to look for cubic tensors
arising from double Lie brackets
with one of the vector fields 
having its value inside the Levi kernel: 
\beq\Label{triple}
	\la \th, [L^3, [L^2, L^1]] \ra, 
	\quad
	\th \in \Om^0,
	\quad L^3, L^2\in H^{10}, 
	\, L^1\in \1{H^{10}}, 
	\quad
	L^1_p \in \1{K^{10}_p}.	
\eeq

However, the following simple example shows that 
\eqref{triple} does not define a tensor in general:

\be\Label{one-ker}
Let $M\subset \C^3_{z_1, z_2, w}$ be
the degenerate quadric
\beq\Label{deg-qua}
	\r = -(w+\w) + z_1\z_1 =0,
\eeq
and consider the $(1,0)$ vector fields
$$
	L^3:= \d_{z_2},
	\quad
	L^2:= \d_{z_2} + cz_2 \1{L^1},
	\quad
	L^1:= \d_{\z_1} + z_1\d_{\w}.
$$
Then 
$$
	[L^3, [L^2, L^1]] = c [L^1, \1{L^1}] = c (\d_{w} - \d_{\w}),
$$
and hence for a contact form $\th\in \Om^0$,
the value $$\la \th, [L^3, [L^2, L^1]] \ra_p$$
depends on $c$,
even though all values $L^j_p$ are independent of $c$.
Note that both $L^2_0$ and $L^3_0$ (but not $\1{L^1_0}$)
are inside the Levi kernel $K^{10}_0$.
Hence the double Lie bracket does not define any tensor
$K^{10}_p\times K^{10}_p \times \1{H^{10}_p}\to \C Q_p$.

The same example also shows that neither
the Levi form derivative $L^3 \la \th, [L^2, L^1]\ra$
considered below
behaves tensorially on the same spaces.
\ee


In contrast, we do get an invariant tensor when {\em both vector fields}
inside the inner Lie bracket have their values in the Levi kernel
at the reference point.
We write $\t^{31}$ for the corresponding tensor, 
reflecting the fact that it will become a restriction 
of the full tensor $\t^3$ below.


\bl\Label{bracket-tensor}
	The double Lie bracket $[L^3, [L^2, L^1]]$ defines
	an invariant tensor
	\beq
		\t^{31}_p\colon \C H_p \times K^{10}_p \times \1{K^{10}_p} \to \C Q_p,
	\eeq
	i.e.\ there exists an unique $\t$ as above satisfying
	$$
		\t^{31}_p(L^3_p, L^2_p, L^1_p) 
		= -i  [L^3, [L^2, L^1]]_p \mod \C H_p
		,
%		\quad \th\in \Om^0,
		\quad L^3, L^2, \1{L^1} \in H^{10},
		\quad L^2_p, \1{L^1_p}\in K^{10}_p.
	$$
	
	Furthermore, $\t^{31}_p$ is symmetric on $K^{10}_p\times K^{10}_p\times \1{K^{10}_p}$ 
	in its $K^{10}$-arguments, 
	and on $\1{K^{10}_p}\times K^{10}_p\times \1{K^{10}_p}$ in its $\1{K^{10}}$-arguments,  
	and satisfies the reality condition
\beq\Label{30-sym}
	\1{\t^{31}_p(v^3, v^2, v^1)} = \t^{31}_p(\1v^3, \1v^1, \1v^2).
\eeq
\el
\bpf
It suffices to show that
\beq\Label{van}
	[L^3, [L^2, L^1]]_p \in \C H_p
\eeq
holds
whenever any of the values $L^j_p$ is $0$.
Since any such $L^j$ can be written as linear combination
$\sum a_k L_k$ with $a_k(p)=0$ and $L_k$ being in the same bundle,
it suffices to assume $L^j = a \2L^j$ with $a(p)=0$,
in which case \eqref{van} is easy to verify.
The reality condition is straightforward and the symmetries follow
from the Jacobi identity.
\epf

A closely related construction 
was proposed by Webster~\cite{W}.



\subsection{$\de$-tensoriality}

\bd
A differential operator $\chi(L^k, \ldots, L^1)$ of order $l$
 is said to be {\em $\de$-tensorial at $p\in M$}  if 
$$
	|\chi(L^k, \ldots,  L^1)_p| \le \de \|j^l_pL^k\| \ldots  \|j^l_pL^1\|
$$
holds whenever $L^j_p=0$ for some $j=1,\ldots, k$.
\ed



\subsection{The Levi form derivative}
As alternative to the double Lie bracket tensor, one can 
differentiate the Levi form after pairing with a contact form,
which is similar to the approach employed by Catlin 
in his boundary system construction:
\beq\Label{der}
	L^3 \la \th, [L^2, L^1] \ra.
\eeq
Again Example~\ref{one-ker} shows that \eqref{der}
does not define a tensor if both $L^2_p$, $L^3_p$
are in the Levi kernel if $L^1_p$ is not.
On the other hand, if {\em both vector fields
$L^1$, $L^2$ inside the Lie brackets
have their value at $p$ contained in the Levi kernel}, we do obtain a tensor
even when the outside vector field $L^3$ is not necessarily contained in $\C H$:


\bl\Label{levi-der}
There exists $c>0$ such that for every $\de\ge0$, the operator
$$
	L^3\la \th, [L^2, L^1] \ra
$$
is $c\de^{1/2}$-tensorial at $p$
for $L^2_p, \1L^1_p \in K^{10}_p(\de)$.

There exists an unique cubic tensor
\beq
	\t^3_p \colon \C T_p \times  K^{10}_p \times \1{K^{10}_p} \to \C Q_p,
\eeq
satisfying
	$$
		\la \th_p, \t^3_p(L^3_p, L^2_p, L^1_p) \ra
		= -i (L^3\la \th, [L^2, L^1] \ra)_p
		,
		\quad \th\in \Om^0,
		\quad L^3 \in \C T,
		\quad L^2, \1{L^1} \in H^{10},
		\quad L^2_p, \1{L^1_p}\in K^{10}_p.
	$$

Furthermore, $\t^3_p$ satisfies the reality condition
\beq\Label{3-sym}
\1{\t^3_p(v^3, v^2, v^1)} = \t^3_p(\1v^3, \1v^1, \1v^2).
\eeq
\el

\bpf
The proof is similar to that of Lemma~\ref{bracket-tensor},
and the symmetry follows directly from the definition.
\epf



\subsection{A normal form of order $3$ and complete set of cubic invariants}
\Label{norm-f}
To compare tensors $\t^{31}$ and $\t^3$,
it is convenient to use a partial normal form
for the cubic terms.
In the sequel we write $\phi_{j_1\ldots j_m}(x^1,\ldots x^m)$
for a polynomial of the multi-degree $(j_1,\ldots, j_m)$
in its corresponding variables.
We also write 
$z_k=(z_{k1},\ldots, z_{km})\in \C^m$
for coordinate vectors and their components.

\bp\Label{3-normal}
For every real hypersurface $M$ in $\C^{n}$ and point $p\in M$ of Levi rank $q$, 
there exist local holomorphic coordinates 
$$
	(w,z)=(w, z_2, z_3)\in \C\times \C^q\times \C^{n-q-1},
$$ 
vanishing at $p$,
where $M$ takes the form
$$
	w + \w = \phi(z,\z, i(w-\w)), 
	\quad
	\phi(z,\z,u) = \phi_2(z,\z,u) + \phi_3(z,\z,u) + O(4), 
$$
where
$$
	\phi_2(z,\z,u) = \phi_{11}(z_2, \z_2),
	\quad
	\phi_3(z,\z,u) = 2\Re \phi_{21}(z,\z_3) + \phi_{111}(z_3, \z_3, u),
$$
and $O(4)$ stands for all terms of total order at least $4$.
\ep


\bpf
It is well-known that the quadratic term $\phi_2$ 
can be transformed into $\phi_{11}(z_2, \z_2)$ 
representing the nondegenerate part of the Levi form.
Furthermore, as customary, we may assume that the cubic term $\phi_3$ has no harmonic terms.

Next, by suitable polynomial transformations 
$$
	(z,w)\mapsto (z + \sum_{j=1}^r z_{2j} h_j(z,w) , w),
$$
we can eliminate all cubic monomials of the form $\z_{2j} h(z,u)$
and their conjugates, where $h(z,w)$ is any holomorphic quadratic monomial.
The proof is completed by inspecting the remaining cubic monomials.
\epf

Next we use the convenient $(1,0)$ vector fields with obvious notation:

\bl\Label{vf-norm}
For a real hypersurface $M\subset \C^{n}$ given by
\beq\Label{graph}
	w + \w = \phi(z,\z, i(w-\w)), \quad (w, z)\in\C\times \C^{n-1},
\eeq
the subbundle $H^{10}$ of $(1,0)$ vector fields is spanned by
$$
	L_j := \d_{z_j} + \frac{\phi_{z_j}}{1-i\phi_u} \d_w, 
	\quad 
	j=1,\ldots, n-1.
$$
More generally, $H^{10}$ is spanned by all vector fields of the form
\beq\Label{lv}
	L_v:= \d_v + \frac{\phi_{v}}{1-i\phi_u} \d_w, 
	\quad v\in \{0\} \times \C^{n-1},
\eeq
where the subscript $v$ denotes the differentiation in the direction of $v$.
%Furthermore, if 
%we assign weights $1>\mu_2\ge \ldots \ge \mu_n$
%to $w, z_2, \ldots z_n$ (with their components and conjugates) respectively
%$\rho = -2\Re w + \phi$ is weighted homogenous 
%with respect to some positive weights
\el



Calculating with special vector fields from Lemma~\ref{vf-norm}, we obtain:
\bc\Label{3-calc}
Let $M$ be in the normal form given by Proposition~\ref{3-normal}.
Then tensors $\t^{31}_p$ and $\t^3_p$ defined in Lemmas~\ref{bracket-tensor}
and ~\ref{levi-der} respectively satisfy
\beq\Label{3-dif}
	\la \th_0,  \t^{31}_0(v^3, v^2, v^1) \ra
	=
	\la \th_0,  \t^3_0(v^3, v^2, v^1) \ra
	= \d_{v^3} \d_{v^2} \d_{v^1} \phi_3,
\eeq
where
$$
	v^3, v^2, \1v^1 \in K^{10}_0 \cong \{0\}\times \{0\} \times \C^{n-q-1},
	\quad
	\th=i \d\rho, 
	\quad
	\rho =-2\Re w + \phi.
$$
Furthermore, the second identity in \eqref{3-dif} still holds for $v^3\in \C H_0$.
%coincide up to suitable constants
%with restrictions of the polarization of $\phi_3$.
\ec
In particular,
$\t^{31}$ is a restriction of $\t^3$ 
to $\C T_p \times  K^{10}_p \times \1{K^{10}_p}$,
explaining the notation.


\br
The term $\phi_{21}$ in Proposition~\ref{3-normal}
represents, up to a nonzero constant multiple,
the cubic invariant tensor 
$\psi_3$ 
introduced by Ebenfelt~\cite{E-jdg}.
It follows from Proposition~\ref{3-normal}
that $\psi_3$ and $\t^3$
coincide (up to a constant) on their common set of definition
and together
constitute the full set of cubic invariants of $M$ at $p$.
%In fact, $\t$ corresponds to a restriction of $\psi_3$,
%whereas $\2\t$ extends that restriction to the full tangent space
%in the first component.
\er



\subsection{Symmetric extensions}
As consequence Corollary~\ref{3-calc},
$\t^3$ is symmetric in
$K^{10}$- or in $\1{K^{10}}$-vectors
whenever two of them occur in any two arguments.
This property leads to a natural symmetric extension:

\bl\Label{3-symm}
The restriction 
$$
	\t^{30}_p \colon \C K_p \times K^{10}_p\times \1{K^{10}} \to \C Q_p
$$ 
of the cubic tensor $\t^3_p$
 admits an unique symmetric extension
$$
	\2\t^{30}_p \colon
	\C K_p \times \C K_p \times \C K_p \to \C Q_p,
$$
satisfying
$$
	\la \th_0,  \2\t^{30}_0(v^3, v^2, v^1) \ra
	= \d_{v^3} \d_{v^2} \d_{v^1} \phi_3,
$$
whenever $M$ is in a normal form $\rho= -2\Re w +\phi=0$
as in Proposition~\ref{3-normal}
and $\th = i\d\rho$.
\el

\br
Note that since $\phi_3$ has no harmonic terms in a normal form,
the extension tensor $\2\t^{30}$ 
always vanishes whenever its arguments
are either all in $K^{10}$ or all in $\1{K^{10}}$.
\er

\be
In contrast to $\t^{30}$, the full cubic tensor $\t^3$
does not in general have any invariant extension to
$\C T\times \C K \times \C K$.
Indeed, consider the cubic $M\subset \C^2$ given by
$$
	\rho := -2\Re w + 2\Re (z^2\z) =0.
$$
Then $\d_{\w} \d_z \d_z\phi_3=0$.
Now consider a change of coordinates with linear part
$(w,z)\mapsto (w, z+ iw)$ transforming $\d_{\w}$
into $\d_{\w}-i\d_{\z}$.
Then, after removing harmonic terms, the new cubic term takes form
$$
	\phi_3 = 2\Re (z^2\z) - 4\Im w z\z.
$$
But then $(\d_{\w} -i\d_{\z}) \d_z \d_z\phi_3\ne 0$, 
i.e.\ the $3$rd derivatives of $\phi_3$ do not transform as tensor
when passing to another normal form.
\ee



%This kind of symmetry can be formalized as follows:
%\bd
%	We call a multilinear map $\psi\colon V_1\times \ldots \times V_m \to W$
%	symmetric if 
%	$$
%		\psi(v_1, \ldots, v_j, \ldots, v_k, \ldots, v_m) 
%		=
%		\psi(v_1, \ldots, v_k, \ldots, v_j, \ldots, v_m)
%	$$
%	holds
%	whenever $v_j, v_k\in V_j\cap V_k$,
%	i.e.\ $\psi$ does not depend on the order of its arguments, whenever the latters can be interchanged.
%\ed
%
%With that definition, we have:
%

%and both extend to a unique cubic tensor
%$$
%	\t^3_p  \colon \C T_p \times \C K_p\times \C K_p \to \C Q_p,
%$$
%which is symmetric in the last two arguments
%and in all $3$ arguments when restricted to
%$(\C K_p)^3$, and
%which vanishes on $(H^{10}_p)^3$ and $(H^{01}_p)^3$
%(i.e.\ has no harmonic parts).
%Furthemore, in the setup of Corollary~\ref{3-calc},
%the extension $\t^3$ still satisfies
%$$
%	\la \th_0,  \t^3_0(v_3, v_2, v_1) \ra
%	= \d_{v_3} \d_{v_2} \d_{v_1} \phi_3.
%$$




%Here $p_{11}$ represents the Levi form and $p_{21}$ and $p_{111}$
%together the invariant cubic tensors
%$$
%	\l_1\colon H^{10}\times H^{10} \times \1{K^{10}} \to \C Q,
%	\quad
%	\l_2\colon \C T\times K^{10}\times \1{K^{10}}\to \C Q.
%$$
%


\subsection{Cubic tensors vanishing for pseudoconvex hypersurfaces}
If $M$ is pseudoconvex, the Levi form $\la \th, [L, \1{L}]\ra$
does not change sign, and therefore the cubic tensor $\t^3$ must vanish identically. We obtain:

\bl\Label{psc-vanish}
Let $M$ be a pseudoconvex hypersurface and $p\in M$.
Then the cubic tensor $\t^3_p$ (and therefore its restriction $\t^{31}_p$) vanishes identically.
Equivalently, the cubic normal form in Proposition~\ref{3-normal} satisfies
\beq\Label{psc-cubic}
	\phi_{21}(z, \z_3) = \sum_{jkl} c_{jkl} z_{2j}z_{2k}\z_{3l},
	\quad
	\phi_{111}(z_3, \z_3, u) =0.
\eeq
\el

The remaining cubic terms in \eqref{psc-cubic} 
can be absorbed into higher weight terms as follows.
We write $o_w(m)$ for terms of weights heigher than $m$.

\bc\Label{pse-nf-cor}
A pseudoconvex hypersurface $M$ in suitable holomorphic coordinates
is given by
\beq\Label{psc-cubic-red}
w + \w = \phi(z,\z, i(w-\w)), 
\quad
\phi =
 \sum_j |z_{2j}|^2 + o_w(1),
\eeq
where $o_w$ is calculated for $w$, $z_{2j}$, $z_{3k}$, and their conjugates, 
having weights 
$1$, $\frac12$, $\frac13$ respectively.
In particular, $M$ cannot be of finite type $3$.
\ec

The last statement follows directly from \eqref{psc-cubic-red},
since the contact orders  with lines in $z_3$ directions
are at least $4$.



\section{Invariant quartic tensors}
If the tensor $\t^3$ (or $\t^{30}$) vanishes,
it is natural to look for higher order invariants by taking iterated Lie brackets
or higher order derivatives of Levi form.
However, 
in contrast to the statements of Lemmas~\ref{bracket-tensor} 
and \ref{levi-der},
we don't obtain a tensor
even when all vector field values are in the kernel in the strongest sense,
as demonstrated by the following example.

\be\Label{ex-4}
Let $M\subset \C^3_{z_1, z_2, w}$ be 
the degenerate quadric 
from Example~\ref{one-ker},
and set 
$$
	L:=\d_{z_2} +  c z_2(\d_{z_1} + \z_1\d_w).
$$
Then 
$$
	[L, \1L] = |c z_2|^2 (\d_{\w} - \d_w),
$$
and
both $\la \th, [L, [\1L, [L, \1L]]] \ra_0$
and $(L\bar L \la \th,  [L, \bar L] \ra)_0 $
depends on $c$, even though
the value $L_0$ is contained in the Levi kernel $K_0^{10}$,
is independent of $c$,
and the cubic tensor $\t^3_0$ identically vanishes.
\ee


\subsection{Vector fields that are in the Levi kernel up to order $1$}
In view of Example~\ref{ex-4},
in order to obtain a tensor, we need to restrict
the choice of the vector fields.
This motivates the following definition:

\bd\Label{ker-1}
Let $L$ be a $(1,0)$ vector field.
We say that 
{\em $L$ is in the Levi $\de$-kernel up to
order $1$ at $p$} if,
it is in the Levi $\de$-kernel at $p$, and if
for any vector fields $L^1\in H^{10}$, $L^2\in \C T$,
and any contact form $\th$, the following holds:
\beq\Label{1-ker}
%	| (\la \th, [L^1, \1L]\ra )_p | \le
%	\de
%	\| L^1_p\| \| \1L_p\| \|\th_p\|,
%	\quad
	|(L^2 \la \th, [L^1, \1L] \ra)_p | \le 
	\de
	\| L^2_p \|
	\| j^1_p L^1\|
	\| j^1_p \1L\|
	\| j^1_p\th\| 
\eeq

More generally, 
for a fixed tangent vector $v\in \C T_p$,
we say that
{\em $L$ is in the Levi kernel up to 
$v$-order $1$ at $p$} 
if \eqref{1-ker}
holds whenver $L^2_p=v$. 
(The latter property obviously depends only on $v$
rather than its vector field extension $L^2$.)
Finally, if the above property holds for all $v$
in a vector subspace $V\subset \C T_p$,
we say that
{\em $L$ is in the Levi kernel up to 
$V$-order $1$ at $p$}.
\ed


%Here the first identity expresses the inclusion $L_p\in K^{10}_p$ in the Levi kernel.
%We emphasize that the order is {\em tangential}
%due to our choice of the differentiation of the Levi form 
%in \eqref{1-ker}
%only along $(1,0)$ and $(0,1)$ vector fields.

It is straightforward to see that for a
\bl\Label{1-jet-dep}
For any $(0,1)$ vector field $\1L$
with $\1L_p\in \1{K^{10}_p}$, 
the expression
$$ 
	(L^2 \la \th, [L^1, \1L] \ra)_p 
$$
only depends on the values $L^2_p, L^1_p$ and $\th_p$,
as well as the $1$-jet of $L$ at $p$.
%In particular,  $L$ being in the Levi kernel up to order $1$
%is a linear condition on the $1$-jet of $L$ at $p$.
%for which the corresponding tensor vanishes.
\el

\be
In the setting of Example~\ref{ex-4},
choosing $L^1:= \d_{z_1}$, we compute
$$
	(L \la \th, [L^1, \1L] \ra )_0 \ne 0,
$$
which shows that $L$ is not in the Levi kernel of tangent order $1$,
even if its value at $0$ is contained in the Levi kernel.
\ee


\br\Label{bracket-restrict}
Using a normal form as in Proposition~\ref{3-normal} 
and calculating with vector fields \eqref{lv},
we can obtain a condition equivalent to
 \eqref{1-ker}
with $L^2$ in $\C H$ (rather than in $\C T$), which can be 
stated in terms of double Lie bracket:
\beq\Label{1-ker'}
	\la \th_p, [L^1, \1L]_p\ra
	= \la \th_p, [L^2, [L^1, \1L]]_p \ra 
	= \la \th_p, [\1L^2 , [L^1, \1L]]_p\ra)
	= 0.
\eeq
\er



A priori, it is not at all clear
that vector fields as in Definition~\ref{ker-1} exist.
The following lemma
provides an easy way of constructing them.


\bl\Label{1-ker-def}
Let $M$ have Levi $\de$-rank $q$ at $p\in M$,
with Levi $\e$-kernel $K^{10}_p(\e)$.
Assume that $L\in H^{10}$ is in the Levi $\e$-kernel
up to order $1$ at $p$,
as per Definition~\ref{ker-1}.
Then $L_p\in K^{10}_p(\e)$ and
\beq\Label{3-vanish}
	| \t^3_p(L^2_p, L^1_p, \1L_p) | 
	\le \e \|L^2_p\| \|L^1_p\| \|\1L_p\| ,
	%= \2\t(\1L^2_p, L^1_p, \1L_p) 
	\quad L^2 \in \C T, \, 
	L^1\in H^{10}, \,
	L^1_p\in K^{10}_p(\e).
\eeq
must hold for all $L^2$, $L^1$.
(Equivalently, $\1L_p$ is contained in the kernel of $\t^3_p$ in the last argument).

Vice versa, assume that $L_p\in K^{10}_p$ and 
\eqref{3-vanish} hold.
Let
$$(\2L^1, \ldots, \2L^q)$$
be a Levi-orthonormal system of 
$(1,0)$ vector fields at $p$,
such that $L$ is Levi-orthogonal to each $\2L^j$, $j=1,\ldots,q$,
in a neighborhood of $p$.
Then $L$ is in the Levi kernel up to order $1$ at $p$.
\el


\bpf
The first part follows directly from the definitions.

Vice versa, since the Levi form has rank $r$ in $p$,
and $L_p$ is Levi-orthogonal to each $\2L^j$,
it follows that $L_p$ is in the Levi kernel, i.e.\
the first expression in \eqref{1-ker} must vanish.

Next, \eqref{3-vanish}
implies that the second expression in \eqref{1-ker}
vanishes whenever $L^1_p\in K^{10}_p$.
Similarly, in view of the symmetry \eqref{3-sym},
also the third expression vanishes
under the same assumption.

Finally, to a general $L^1$ with $L^1_p\notin K^{10}_p$,
we can always add a linear combination of $\2L^j$
to achieve the inclusion of the value at $p$ in the Levi kernel.
Since $L$ is Levi-orthogonal to each $\2L^j$ identically in a neighborhood of $p$,
this does not change \eqref{1-ker},
completing the proof.
\epf

In particular, in view of Lemma~\ref{psc-vanish} we obtain:

\bc\Label{gen-pt}
Let $M$ be pseudoconvex.
Then every $v\in K^{10}_p$
extends to a $(1,0)$ vector field,
which is in the Levi kernel up to order $1$ at $p$.
\ec

\br
More generally, a similar result
can be obtained without pseudoconvexity
for a $v\in K^{10}_p$ 
whose conjugate $\1v$ is in the (right) kernel of $\t^3$, 
i.e.\ satisfying 
$$
	\t^3_p(L^3_p, L^2_p, \1v) = 0
$$
for all $L^3$, $L^2$.
Then there exists a $(0,1)$ vector field $\1L$ extension of  $\1v$,
which is in the Levi kernel up to order $1$ at $p$.
\er



\subsection{Invariant submodule sheaves of vector fields}
\Label{submodules}
%To better formalise the underlying structure
The notion of the {\em Levi kernel inclusion up to order $1$} 
has been defined pointwise in Definition~\ref{ker-1}.
In order to have a uniform control for 
Levi kernels in nearby points,
we shall need to define
corresponding sheafs of vector field submodules as follows. 
%Consider the sheaf of all smooth vector fields regarded
%as a module over the sheaf of all smooth functions.

\bd\Label{q-sheaf}
Let $M\subset \C^n$ be a real hypersurface.
Denote by $\6T^{10}$ the sheaf of all 
 $(1,0)$ vector fields on $M$.
For every $q\le n-1$,
define $\2{\6S}^{10}(q)\subset \6T^{10}$ to be the submodule sheaf
consisting of all vector fields on $M$ which are
contained in the Levi kernel up to order $1$
 at every point of Levi rank $\le q$.
\ed

As a direct consequence of Lemma~\ref{1-ker-def},
we obtain the following strengthening of Corollary~\ref{gen-pt}:
\bc\Label{q-sheaf-sect}
Let $M$ be a pseudoconvex hypersurface.
%and $p\in M$ a point of Levi rank $q$.
Then for every $q$, local sections of $\2{\6S}^{10}(q)$
%vanish at all points of Levi rank $\le q$ and
span the Levi kernel $K^{10}_x$
at every point $x\in M$ of Levi rank $q$.
\ec

Note that to guarantee the existence of
sufficiently many sections as in 
Corollary~\ref{q-sheaf-sect},
 it is important to restrict
the property underlying Definition~\ref{q-sheaf}
only to points of Levi rank $\le q$.
Without that restriction, 
the sheaf would become trivial
e.g.\ for any
manifold $M$ that is generically 
Levi-nondegenerate
(which is the case for any $M$ of finite type).

Definition~\ref{q-sheaf}
requires to check the condition at eery point of Levi rank $\le q$,
which may be difficult to deal with in practice.
Analysing the proof we arrive at the smaller submodule sheaf
$\6S^{10}(q)$ (as defined in Theorem~\ref{main}, part (6)):

\bd\Label{vanish-level}
Define the submodule sheaf $\6S^{10}(q)$
to be generated by all $(1,0)$ vector fields $L$
satisfying $L\in V_L^\perp$,
with $V_L\subset H^{10}$
a Levi-nondegenerate subbundle
of rank $q$ in a neighborhood of $p$.
\ed

Then Lemma~\ref{1-ker-def} yelds all the needed properties:

\bc\Label{span-ker}
Under the same assumptions as in Corollary~\ref{q-sheaf-sect},
for every $q$,
$\6S^{10}(q)$ is a submodule sheaf of $\2{\6S}^{10}(q)$,
which has its local sections also
span the Levi kernel $K^{10}_x$
at every point $x\in M$ of Levi rank $q$.
\ec



%\section{}

\subsection{Construction of the quartic tensor}
\Label{quartic}
%Since $\2\t$ is obtained by differentiating the Levi form (up to a constant),
%we shall also write $D\l_p:=\2\t$.
Equipped with special vector fields as in Definition~\ref{ker-1},
we can now define an invariant quartic tensor
via second order derivatives of the Levi form:

\bl\Label{d2}
Let $M$ be such that the cubic tensor $\t^3_p$ vanishes for some $p\in M$.
Then there exists an unique tensor
$$
	\t^4_p \colon  \C T_p \times \C T_p 
	\times K^{10}_p\times \1{K^{10}_p} \to \C Q_p,
$$
such that for any $(1,0)$ vector fields
$\1{L^1}, L^2\in H^{10}$ that are in the Levi kernel
up to order $1$ at $p$,
any vector fields $L^3, L^4\in \C T$,
and any contact form $\th\in \Om^0$,
\beq\Label{1-levi'}
	\la \th_p, \t^4_p(L^4_p, L^3_p, L^2_p, L^1_p)\ra 
	= -i(L^4 L^3 \la \th, [L^2, L^1]\ra)_p.
\eeq

More generally, \eqref{1-levi'} still holds
whenever both $\1L^1$ and $L^2$
are in the Levi kernel
up to $L^j_p$-order $1$ at $p$, for $j=3,4$.
\el

\bpf
Similar to the proof of Lemma~\ref{bracket-tensor},
it suffices to prove that the right-hand side of \eqref{1-levi'}
vanishes whenever
either $L^k=a\2L^k$ for some $k=1,2,3,4$, 
or $\th = a\2\th$, where $a$ is a smooth function vanishing at $p$.
In the following $\2a$ will denote either $a$ or the conjugate $\1a$
and we assume (without loss of generality)
that each of $L^3, L^4$
is contained in either $H^{10}$ or $H^{01}$.

Now the vanishing of the right-hand side in \eqref{1-levi'} 
is obvious for $k=4$.
For $k=3$, it takes
the form
$$
	(L^4\2a)_p (\2L^3 \la \th, [L^2, L^1] \ra)_p,	
$$
which must vanish in view of Definition~\ref{ker-1}.
For $k=1$, we obtain
$$
	(L^4 \bar a)_p (L^3 \la \th, [L^2, \2L^1] \ra)_p
	+ 	(L^3 \bar a)_p (L^4 \la \th, [L^2, \2L^1] \ra)_p
	+ 	(L^4 L^3 \bar a)_p (\la \th, [L^2, \2L^1] \ra)_p
	,
$$
which again vanishes in view of Definition~\ref{ker-1}.
For $k=2$, the proof follows from the case $k=1$
by exchanging $L^2$ and $L^1$ and conjugating.
Finally, for $\th = a \2\th$, we obtain
$$
	(L^4 a)_p (L^3 \la \2\th, [L^2, L^1] \ra)_p
	+ 	(L^3 a)_p (L^4 \la \2\th, [L^2, L^1] \ra)_p
	+ 	(L^4 L^3 a)_p (\la \2\th, [L^2, L^1] \ra)_p,
$$
which vanishes by the same argument.
\epf


\br
In higher generality, when the cubic tensor $\t^3_p$ may not vanish completely,
a quartic tensor $\t^4_p$ can still be constructed via \eqref{1-levi'} 
along certain kernels of $\t^3_p$.
We will not pursue this direction as our focus here is on 
the pseudoconvex case when $\t^3_p$ always vanishes identically.
\er







\subsection{A normal form up to weight $1/4$}
Since the cubic normal form for pseudoconvex hypersurfaces \eqref{psc-cubic-red}
is in some sense lacking nondegenerate terms, 
we extend it by lowering the weight of $z_3$ from $1/3$ to $1/4$
(an renaming $z_3$ to $z_4$):

\bp\Label{4-normal}
For every pseudoconvex real hypersurface $M$ in $\C^{n}$ 
and point $p\in M$ of Levi rank $q$, 
there exist local holomorphic coordinates 
$$
	(w,z)=(w,z_2, z_3)\in \C\times \C^q\times \C^{n-q-1},
$$ 
vanishing at $p$,
where $M$ takes the form
\beq\Label{phi24}
	w+ \w = \phi(z,\z , i(w-\w)), 
	\quad
	\phi = \phi_2 + \phi_4 + o_w(1), 
\eeq
where
$$
	\phi_2(z,\z,u) = \sum_{j=1}^q |z_{2j}|^2,
	\quad
	\phi_4(z,\z,u) = 2\Re \phi_{31}(z_4,\z_4) + \phi_{22}(z_4,\z_4),
$$
such that the weight estimate $o_w$ is calculated for $u$, $z_{2j}$, $z_{4k}$, and their conjugates, 
having weights 
$1$, $\frac12$, $\frac14$ respectively.
Here each polynomial $\phi_{jk}$ is bihomogenous of bidegree $(j,k)$ in its arguments.

Furthermore, the following hold:
\begin{enumerate}
\item
For every 
$v\in K^{10}_0\cong \{0\} \times \{0\} \times \C^{n-q-1}$, 
the vector field $L_v$ given by \eqref{lv} is 
in the Levi kernel up to $v^0$-order $1$ at $0$
for any $v^0\in \C K_0$.
\item
For $v^4, v^3\in \C K_0$
and
$v^2, \1v^1\in K^{10}_0$,
we have
\beq\Label{phi-diff}
	\t^4_p(v^4, v^3, v^2, v^1) 
	= \d_{v^4} \d_{v^3} \d_{v^2} \d_{v^1} \phi_4.
\eeq
%
%Finally, the polarization of $\phi_4$ defines 
%an invariant tensor 
%$$(\C K_p)^4 \to \C Q_p, $$
%whose restriction to $$(\C K_p)^2 \times K^{10}_p\times \1{K^{10}_p}$$
%coincides up a nonzero constant with corresponding restriction of $D^2\l_p$.
In particular, the restriction
\beq\Label{t40}
	\t^{40}_p \colon \C K_p\times  \C K_p \times K^{10}_p\times \1{K^{10}_p} \to \C Q_p
\eeq 
of $\t^4_p$ is symmetric whenever its arguments can be interchanged
and satisfies the reality condition
$$
	\1{\t^4_p(v^4, v^3, v^2, v^1)}
	=
	\t^4_p(\1v^4, \1v^3, \1v^1, \1v^2).
$$
\end{enumerate}
\ep

\bpf
The existence of the desired normal form 
is a direct consequence of Lemma~\ref{psc-vanish}.
A direct calculation shows the special vector fields in \eqref{lv}
with $v\in \{0\} \times \{0\} \times \C^{n-q-1}$ are in the Levi kernel up to tangential order $1$
as claimed.
The remaining properties are straightforward. 
\epf


Similarly to Corollary~\ref{3-calc},
one can also show the following quartic Lie bracket
representation of the restriction 
\eqref{t40}:
\bl
The restriction $\t^{40}_q$
%$$
%	\t^{40}_p \colon  \C H_p \times \C H_p 
%	\times K^{10}_p\times \1{K^{10}_p} \to \C Q_p
%$$
satisfies
$$
%	\beq\Label{1-levi'}
	\la \th_p, \t^{41}_p(L^4_p, L^3_p, L^2_p, L^1_p)\ra 
	= -i\la \th_p, [L^4, [L^3, [L^2, L^1]]]_p\ra
%\eeq
$$
whenever $L^2, \1L^1\in H^{10}$ are in the Levi kernel
up to $\C K$-order $1$ at $p$, $L^3, L^4\in \C H$,
and $\th\in \Om^0$ is any contact form.
\el


\br
It is easy to see that pseudoconvexity of $M$ implies that the quartic polynomial 
$\phi_4(z_3,\z_3)$ in \eqref{phi24} is plurisubharmonic. 
Conversely, every plurisubharmonic $\phi_4$ appears
in a normal form of some pseudoconvex hypersurface,
e.g.\ the model hypersurface 
$$
	w+ \w = \sum_{j=1}^q |z_{2j}|^2  + \phi_4(z_4, \z_4).
$$
%It is, however, not true that 
\er

\subsection{Symmetric extension}
Similarly to Lemma~\ref{3-symm}, we obtain a symmetric extension
for the Levi kernel restriction of $\t^4$:
\bl\Label{4-symm}
The restriction 
$$
	\t^{40}_p \colon \C K_p\times  \C K_p \times K^{10}_p\times \1{K^{10}_p} \to \C Q_p
$$ 
of the quartic tensor $\t^4_p$
 admits an unique symmetric extension
$$
	\2\t^{40}_p \colon
	\C K_p \times \C K_p\times \C K_p \times \C K_p \to \C Q_p,
$$
satisfying
\beq\Label{tensor-dif}
	\la \th_0,  \2\t^{40}_0(v^4, v^3, v^2, v^1) \ra
	= \d_{v^4} \d_{v^3} \d_{v^2} \d_{v^1} \phi_4,
\eeq
whenever $M$ is in a normal form $\rho= -2\Re w +\phi=0$
as in Proposition~\ref{4-normal}
and $\th = i\d\rho$.
In fact, \eqref{tensor-dif}
holds whenever
$\phi$
satisfies $d\phi_0=0$
and 
$ \d_{v^j} \d_{v^2} \d_{v^1} \phi_3=0$
for $j=3,4$.
\el




\section{Applications and properties of the quartic tensor}

\subsection{Relation with the D'Angelo finite type}
The quartic tensor $\t^4$ can be used to completely characterize
the finite type up to $4$ in the sense of D'Angelo~\cite{D82}
(the last property is related to D'Angelo's ``Property P'', see \cite[Definition~5.1]{D82}):
%We shall use the following standard lemma:
%\bl
%Let $p(z,\z)$ be a homogeneous polynomial of degree $4$ in $(z, \z)\in \C\times \C$.
%Suppose $p$ is plurisubharmonic.
%Then 
%$$
%	p(z,\bar z) + \Re h(z) \ge 0
%$$
%holds for a suitable holomorphic polynomial $h$ of degree $4$.
%\el
%\bpf
%
%\epf

\bp\Label{type-quartic}
Let $M$ be a pseudoconvex hypersurface
with nontrivial Levi kernel at $p$.
Then $M$ is of D'Angelo type $4$ at $p$
if and only if for every nonzero vector $v\in K^{10}_p$, the tensor $\t^4_p$
does not vanish when restricted to 
\beq\Label{v-res}
	(\C v + \C \1v) \times (\C v + \C \1v) \times \C v\times \C \1v.
\eeq
In fact, the latter property implies the following stronger nonvanishing conclusion:
$$
	\t^4_p(v,\1v, v, \1v) \ne 0.
$$
\ep

\bpf
We may assume $M$ is put into its normal form as in Proposition~\ref{4-normal}.

If the restriction of $\t^4_p$ vanishes on \eqref{v-res} for some $v\ne 0$,
we may assume $v=\d_{z_{31}}$, where $z_3=(z_{31},\ldots, z_{3,n-r})$.
Then it follows from the normal form that the line $\C v$ has order of contact with $M$
higher than $4$, hence the D'Angelo type at $p$ is also higher than $4$.

On the other hand, suppose the restriction of $\t^4_p$ to \eqref{v-res} does not vanish for any 
$v\ne 0$. Assume by contradiction, 
there exists a nontrivial holomorphic curve 
$$
	\g\colon (\C,0) \to (\C^{n+1},0),
	\quad
	\g(t) = \sum_{k\ge k_0} a_k t^k,
	\quad 
	a_{k_0}\ne 0,
$$
whose contact order with $M$ at $0$ is higher than $4$.
Recall that the contact order is given by
$$
	\frac{\nu(\rho\circ\g)}{\nu(\g)},
$$
where $\rho$ is any defining function of $M$ and $\nu$ is the vanishing order at $0$,
in particular, $\nu(\g) = k_0\ge 1$.
Taking $\rho := -2\Re w +\phi$, we must have $a_{k_0}\in \{0\} \times \C^n$,
otherwise the contact order would be $1$.
Similarly, expanding $\rho\circ \g$, it follows by induction that 
$$
	a_{l}\in \{0\} \times \C^{n},
	\quad
	l< 4k_0,
$$
and
$$
	a_{k}\in \{0\} \times \{0\} \times \C^{n-r},
	\quad
	k< 2k_0,
$$
for otherwise the contact order would be less than $4$.
Finally collecting terms of order $4k_0$
and using our assumption that the contact order is greater than $4$,
we obtain
\beq\Label{4k}
 	\phi_2(a_{2k_0} t^{2k_0}, \1a_{2k_0} \1t^{2k_0} ) + \phi_4(a_{k_0} t^{k_0}, \1a_{k_0}\1t^{k_0}) =0.
\eeq
In particular, it follows that 
$$
	\phi_4(a_{k_0} \xi, \1a_{k_0}\1\xi) = c\xi^2\bar\xi^2.
$$
Since $\phi_4$ is plurisubharmonic, we must have $c\ge 0$.
Hence both terms in \eqref{4k} are nonnegative,
and therefore must vanish.
In particular, 
$
	c a_{k_0}^2 \1a_{k_0}^2 = 0,
$
implying 
$$
	\phi_4(a_{k_0} \xi, \1a_{k_0}\1\xi) = 0,
$$
which is in contradiction with our nonvanishing assumption on $\t^4_p$.
Hence the D'Angelo type is $4$ completing the proof of the converse direction.

Finally, the last statement follows from the plurisubharmonicity of $\phi_4$
in any normal form.
\epf


The pseudoconvexity assumption in Proposition~\ref{type-quartic} cannot be dropped:

\be
Let $M\subset \C^3_{w, z_1, z_2}$ be given by
$$
	2\Re w = |z_1|^2 - |z_2|^4.
$$
Then $M$ contains the image of the curve $t\mapsto (0, t^2, t)$ 
and is hence of infinite type at $0$.
On the other hand, $M$ is in the normal form \eqref{phi24}
and hence $\t^4_0(v, \1v, v, \1v)\ne 0$  for any $v\ne0\in K^{10}_0$.
\ee

\subsection{Uniformity of the quartic tensor}
The sheaves  $\6S^{10}(q)$
introduced in Definition~\ref{vanish-level}
 can be used to 
obtain a uniform behavior of 
$\t^4_p$ as $p$ varies
over the set of nearby points
of bounded Levi rank.
In fact, as direct consequence
from the definition and Corollary~\ref{span-ker}, 
we obtain
that  $\t^4_p$
can be calculated using local sections of $\6S^{10}(q)$:
\bc\Label{uni-t4}
For every 
vector fields $L^4, L^3\in \C T$
and $L^2, \1L^1\in \6S^{10}(q)$
defined in an open set $U\subset M$,
the identity 
\eqref{1-levi'} holds
simultaneously
for all points $p\in U$
of Levi rank $q$.
\ec


\br
In the context of Corollary~\ref{uni-t4},
it is essential to require the vector fields $L^2, L^1$
to be contained in Levi kernels up order $1$ (rather than merely contained there).
In fact, for a higher order perturbation of Examples~\ref{ex-4} where 
$0$ is the only Levi-degenerate point, 
choosing vector fields $L^j$ as higher order perturbations of
the vector field $L$ in the example or its conjugate
would violate \eqref{1-levi'}.
\er


It is important to note that the conclusion of Corollary~\ref{uni-t4}
may not hold for points $p\in U$ of Levi rank $>q$
when $L^2_p, \1L^1_p\in K^{10}_p$.
In fact, $\t^4_p$ may not even be continuous 
e.g. may vanish for $p$ of higher Levi rank even when $\t^4_{p_0}$ does not vanish 
on any line for $p_0$ of Levi rank $q$.
This is illustrated by D'Angelo's celebrated example 
where the finite type is not upper-semicontinuous \cite{D80, D82}:


\be[J. P. D'Angelo]\Label{da-ex}
Let $M\subset\C^3_{w, z_1, z_2}$ be given by
$$
	2\Re w = |z_1^2 - w z_2|^2 + |z_2|^4.
$$
Then $M$ is of Levi rank $0$ and finite type $4$ at $0$ and hence $\t^4_0$ does not vanish on the lines 
products \eqref{v-res} in view of Proposition~\ref{type-quartic}.
In fact, $M$ is in its normal form as in Proposition~\ref{4-normal} with
$\phi_4 = |z_1|^4 + |z_2|^4$, and hence 
$$
	\t^4_0(v, \1v, v, \1v) = 4(|v_1|^4 + |v_2|^4),
	\quad v\in K^{10}_0 \cong \{0\} \times C^2_{z_1, z_2}.
$$

On the other hand, at every $p=(it, 0, 0)$ on the imaginary axis with $t\ne 0$,
the Levi rank is $1$, and $M$ can be locally transformed into a normal form 
\eqref{phi24} with vanishing $\phi_4$ implying $\t^4_p(v, \1v, v, \1v)=0$
for any $v\in K^{10}_p$. Thus $\t^4_p(v, \1v, v, \1v)$ cannot be continuous
for any $v=v(p)$ converging to any $v(0)\ne 0$
as $p\to 0$.

Of course, this phenomenon is closely related to the lack of upper-semicontinuity of the type
demonstrated by D'Angelo.
\ee



\subsection{Kernels of quartic tensors}
For any homogenous polynomial, consider the following notion of
holomorphic kernel:
\bd
	The {\em holomorphic kernel of a homogeneous polynomial} 
	$P(z,\bar z)$, $z\in \C^n$, 
	is defined to be the subspace of all $(1,0)$ vectors $v$ such that
	\beq\Label{kernel-def}
		\d_v P(z,\z) \equiv \d_{\1v}P(z,\z) \equiv 0.
	\eeq
	Equivalently, the holomorphic kernel is the space of all $v$
	such that both $v$ and $\1v$ belong to the kernel 
	of the polarization of $p$.
%	The $(0,1)$ kernel is the conjugate of the $(1,0)$ kernel,
%	and the {\em (total) kernel} is the sum of both $(1,0)$ and $(0,1)$ kernels.
\ed
It is straightforward to see the following simple characterization of the kernel:
\bl\Label{ker-coor}
	The holomorphic kernel of $p$ is the maximal subspace $V$ such that,
	there exists a linear change of coordinates such that
	$$V= \oplus_{j=1}^l (\C \d_{z_j} \oplus \C \d_{\z_j})$$
	and $P(z,\z)$ is independent of the variables $z_1,\ldots, z_l$ and their conjugates.
\el

\bd\Label{rank}
	The {\em rank} of $P$ is $n-d$, where $d$ is the dimension of the holomorphic kernel.
\ed

Also separating bihomogeneous components in \eqref{kernel-def}, we obtain:
\bl
	Let 
	$$
		P(z,\z)=\sum P_{kl}(z,\z)
	$$ 
	be a decomposition into components $P_{kl}$ of bidegree $(k,l)$ in $(z,\z)$.
	Then the holomorphic kernel of $P$ equals the intersection of kernels of $P_{kl}$ for all $k,l$.
\el

Next we compare the holomorphic kernel of the polynomial $\phi_4$
in the normal form given by Proposition~\ref{4-normal}
and the restriction 
$$
	\t^{40}_p \colon \C K_p \times \C K_p \times K^{10}_p \times \1{K^{10}_p} \to \C Q_p.
$$
of the quartic tensor $\t^4_p$.

\bd
The {\em holomorphic kernel} of $\t^{40}_p$ is 
$V\cap \1V$, where
$$
	V = \ker \t^{40}_p = \{ v\in \C K_p : 
		\t^{40}_p(v, v^3, v^2, v^1) =0
		\text{ for all }
		v^3, v^2, v^1\}.
$$
\ed

First of all, remark that 
without pseudoconvexity assumption,
the holomorphic kernel of $\t^{40}_p$
may get larger than that of $\phi_4$:
\be\Label{diff-kernels}
Let $M\subset \C^3_{w, z_1, z_2}$ be given by
$$
	2\Re w = \phi_4(z,\z):= 2\Re (z_1^3 \z_2).
$$
Then the arguments in the proof of Proposition~\ref{4-normal}
can be used to show that \eqref{phi-diff} still holds,
%i.e.\
%$$
%	\t^{40}_0 = c dz_3 \otimes dz_3 \otimes d_z
%$$
implying that $\d_{z_2}$ and $\d_{\z_2}$
are in the kernel of $\t^{40}_0$
in the $1$st and $2$nd arguments
but not in the $3$rd one.
\ee

On the other hand, in presence of pseudoconvexity,
both kernels must coincide as the following lemma shows.
As a matter of convention, for a multilinear function 
$f(v^1, \ldots, v^m)$, we call its kernel in the $k$th argument
the space of all $v^k$ such that $f(v^1, \ldots, v^m)=0$ holds
for all $v^j$ with $j\ne k$.

\bl\Label{ker-rel}
	Let $M$ be in its normal form given by Proposition~\ref{4-normal},
	and
	assume that $M$ is pseudoconvex.
	Then both holomorphic kernels of $\t^{40}$ in the $1$st and $2$nd arguments coincide with holomorphic kernel 
	$V$ of $\phi_4$.
	Furthermore,
	the kernels of $\t^{40}$ in the $3$rd and $4$th arguments
	 coincide respectively with $V$ and $\1V$.
\el


\bpf
As direct consequence of \eqref{phi-diff} we obtain
that the holomorphic kernel of $\phi_4$ is contained
in the kernel of $\t^{40}_p$ in each argument.

Vice versa, let $v$ be $(1,0)$ vector in the holomoprhic 
kernel of $\t^{40}_p$ (in the $1$st argument).
We write $\xi=z_4$ for brevity.
After a linear change of coordinates we may assume 
$v=\d_{\xi_1}$, where $\xi_1$ is the first component of $\xi$
in the notation of Proposition~\ref{4-normal}.
Then it follows from \eqref{phi-diff} that
$\d_{\xi_1} \phi_4$ is harmonic.
Since $\phi_4$ has no harmonic terms, it must have the form
$$
	\phi_4 = 2\Re (\bar\xi_1 h(\xi)) + R,
$$
where $h$ is holomorphic and $R$ is independent of $\xi_1$.
Now since $M$ is pseudoconvex, $\phi_4$ is plurisubharmonic,
in particular,
\beq\Label{t}
	(\d_{\xi_1} + t \d_{\xi_j}) (\d_{\bar\xi_1} + t \d_{\bar\xi_j}) 
	\phi_4 \ge 0
\eeq
holds for all $t\in \R$.
Then for $t=0$, we obtain $\d_{\xi_1}h\ge 0$.
Since $h$ is holomorphic, we must have $\d_{\xi_1}h \equiv 0$.
Hence the linear part of \eqref{t} must be $\ge 0$
and therefore equal to $0$, since $t$ is any real number.
But this means $h\equiv 0$, and hence $v=\d_{\xi_1}$ is in the 
holomorphic kernel of $\phi_4$ as claimed.

The claimed statements for kernels in other arguments of $\t^{40}_p$
are obtained by 
repeating the same proof. 
\epf

In view of Lemma~\ref{ker-rel},
we simply refer to the {\em holomorphc kernel of $\t^{40}$}
for its kernel in the $1$st (and, equivalently, in the $2$nd) argument,
%and to  {\em $(1,0)$ and $(0,1)$ kernels of $\t^{40}$} for 
%its $(1,0)$ and $(0,1)$ components.
Also the {\em rank of $\t^{40}$} is 
$\dim K^{10}_p - d$,
where $d$ is the dimension of its holomorphic kernel,
which coincides with the rank of $\phi_4$
in the sense of Definition~\ref{rank}.



\subsection{Relation with Catlin's multitype}
\Label{m-type}
Recall that the multitype of $M\subset \C^{n+1}$ at $p$
is the lexicographically maximal
$$
	\L = (\l_1, \ldots, \l_{n+1}), 
	\quad
	\l_1 \ge \ldots \ge \l_{n+1},
$$
such that a defining function $\rho$ of $M$ 
satisfies 
\beq\Label{rho-wt}
	\rho = O_w(1)
\eeq
for a choice of holomorphic coordinates 
$(z_1, \ldots, z_{n+1})$,
which together with their conjugates are assigned 
respectively the weights 
$(\l_1^{-1}, \ldots, \l_{n+1}^{-1})$.

The main problem with multitype
is that for a given coordinate representation,
it is difficult to know whether concrete weights
actually realize their lexicographic maximum.
We now give a simple way of calculating a part 
of the multitype in terms of the rank of the tensor $\t_4$.
In case $\t_4$ has trivial kernel, that gives the complete multitype.

\bp\Label{multi-quartic}
Let $M\subset \C^{n}$ be a pseudoconvex hypersurface,
and $p\in M$ a point with Levi form of rank $q_2$ 
and the restricted quartic tensor $\t^{40}$ of rank $q_4$.
Then the multitype $\L = (\l_1, \ldots, \l_{n})$ of $M$ at $p$
satisfies
\beq\Label{wt-eq}
	\l_1 = 1, 
	\quad
	\l_2 = \ldots \l_{q_2+1} =2, 
	\quad
	\l_{q_2+2} = \ldots = \l_{q_2 + q_4 +1} = 4,
\eeq
and 	
\beq\Label{wt-ineq}
	\l_k > 4, \quad k> q_2 + q_4 +1.
\eeq

In particular, if $\t^{40}$ has only trivial kernel,
the multitype is
$(1, 2,\ldots, 2, 4, \ldots, 4)$,
where the number of $2$'s equals the Levi rank.
\ep

\bpf
By Lemma~\ref{ker-coor}, 
in addition to the normal form in Proposition~\ref{4-normal},
we can make $\phi_4$ independent of the last $d$ coordinates,
where $d$ is the dimension of the $(1,0)$ kernel of $\t_4$.
This shows that it is possible to achieve 
\eqref{rho-wt}
with weights
satisfying both \eqref{wt-eq} and \eqref{wt-ineq}.

The actual multitype may only be lexicographically higher,
in particular, \eqref{wt-ineq} is already satisfied.
Assume by contradiction that we have another choice of coordinates 
with higher weights failing one of the equalities in \eqref{wt-eq}.
However, we must obviously have $\l_1=1$
and the Levi form invariance forces the next $q_2$ weights to be equal $2$.
Therefore we must have some $\l_k > 4$ for $k\le q_2+q_4+1$.
In those coordinates, we would have the same normal form as in Proposition~\ref{4-normal}
with $\phi_4$ being independent of $z_j$ at least for $j\ge q_2 + q_4 +1$.
That, however, would mean that the rank of $\t_4$ is less than $q_4$, which is a contradiction.
Hence the multitype must satisfy all of \eqref{wt-eq} as claimed.
\epf




\section{Ideal sheaves for Levi rank level sets}\Label{ideal}

We use the vector field submodule 
sheaves $\6S^{10}(q)$ in Definition~\ref{vanish-level}
to define invariant ideal sheaves of smooth functions 
$\6I(q)$ (as in Theorem~\ref{main}, part (5)):
%vanishing on the Levi rank level sets:


%
%\bd
%Define $q$th {\em sublevel ideal sheaf} $\6I(q)$
%to be the ideal sheaf generated by 
%%real and imaginary parts of the 
%functions of the form
%$$
%	f_{L^3, L^2, L^1, \th}:=  L^3 \la \th, [L^2, L^1] \ra,
%	\quad
%	L^3\in \C T, 
%	\quad
%	L^2, \1L^1 \in \6S^{10}(q),
%	\quad
%	\th \in \Om^0.
%$$
%\ed
%
%
%
%Using sections of $\6S^{10}(q)$ we define
%the invariant ideal sheaf 
%$\6I(q)$ (as in Theorem~\ref{main}, part (5)):

\bd
Let $M\subset\cn$ be a pseudoconvex hypersurface.
For every $q$,
define $\6I(q)$ to be the ideal sheaf 
generated by all (smooth complex) functions $f$ of the form
$$
	f = L^3 \la \th, [L^2, L^1] \ra,
	\quad 
$$
where $\th\in\Om^0$ is a contact form, 
$L^3\in \C T$ arbitrary complex vector field,
and $L^2, \1L^1 \in \6S^{10}(q)$
arbitrary sections.
\ed

As direct consequence of Corollary~\ref{span-ker} and 
Lemma~\ref{psc-vanish}, we obtain
a general way of constructing submanifolds
containing level sets of Levi rank:

\bc\Label{iq-van}
Let $M$ be a pseudoconvex hypersurface.
Then every local section in $\6I(q)$
vanishes at all points of Levi rank $q$.
In particular, for any collection $f^1, \ldots, f^m$
of real functions from the real part $\Re \6I(q)$
defined in an open set $U\subset M$
satisfying 
$$
	df^1 \wedge \ldots \wedge df^m\ne 0,
$$
the submanifold
$$
	S = \{ f^1 = \ldots = f^m =0 \}
$$
contains the set of all points of Levi rank $q$ in $U$.
\ec

\br
Note that due to our definition of $\6I(q)$,
any complex multiple of a local section is again a local section.
Consequently, it suffices to take only sections in $\Re \6I(q)$
to define the same set.
\er


We next apply the quartic tensor to describe the differentials
of sections in $\6I(q)$.

\bd
For an ideal sheaf $\6I$ define it {\em kernel} at $p$
$$
	\ker_p \6I \subset \C T_p,
$$
to be the intersection of kernels of all differentials $df_p$,
where $f$ is any local section of $\6I$ in a neighborhood of $p$.
\ed

Then Corollary~\ref{iq-van} implies:

\bl
Let $p\in M$ be a point of Levi rank $q$.
Then the kernel of the $q$th sublevel ideal $\6I(q)$ at $p$
coincides with the kernel of the quartic tensor $\t^4_p$.
\el

We can now summarize this paragraph's results as follows:

\bp\Label{main0}
Let $M\subset\cn$ be a pseudoconvex real hypersurface,
and $p\in M$ a point of Levi rank $q$.
%and the restricted quartic tensor $\t^{40}_p$
%of corank $q_4$.
Then in a neighborhood of $p$, 
the set 
%$$\{\rk \l^2_x $$
 of all points of the same Levi rank $q$
is contained in a real submanifold $S\subset M$
%of holomorphic dimension at most $q_4$ 
through $p$ such that
%whose tangent space at $p$
%coincides with the kernel of 
$$
	T_p S = \ker \t^4_p,
$$
and $S$ is given by the vanishing 
of local sections
$$
	f^1, \ldots, f^m \in \6I(q),
	\quad
	df^1\wedge \ldots \wedge df^m \ne 0.
$$
%
%Suppose that the multitype at $p\in M$,
%$\6M(p ) = (\l_1, \ldots, \l_n)$  
%satisfies $\l_j\le 4$ for $j\le n-q$.
In particular, when $M$ of finite type $4$ at $p$,
the intersection of $T_pS$
with the Levi kernel at $p$ is totally real.
\ep

%
%\bc
%In the setting of Theorem~\ref{main},
%suppose that the multitype
%$$
%	\6M(p ) = (m_1, \ldots, m_n)
%$$ 
%satisfies $m_l \le 4$ for some $l$.
%Then 
%\ec
%


%\bc
%Let $M\subset\cn$ be a pseudoconvex real hypersurface
%of finite type $4$.
%Then $M$ is a disjoint union of CR submanifolds
%$$
%	M = M_0 \cup \ldots M_{n-1}
%$$
%such that each $M_q$ 
%\ec



\section{Relation with Catlin's boundary systems}\Label{bs}
\subsection{Maximal Levi-nondegenerate subbundle}


%\subsection{Relation with Catlin's boundary systems}
%Let $M$ be pseudoconvex with Levi rank $q$ at $p\in M$.

Recall that {\em Catlin's boundary system} construction
for a hypersurface $M$ at a point $p\in M$
begins with a maximal collection of $(1,0)$ vector fields $L_2, \ldots, L_{q+1}$ tangent to $M$
such that the Levi form matrix 
$$
	(\la \th,  [L_j, \1L_k] \ra)_{2\le j,k \le q}
$$
is nonsingular. In particular, $q$ must be equal to the Levi rank at $p$.

Invariantly, consider any {\em maximal Levi-nondegenerate subbundle} through $p$,
i.e.\ any smooth subbundle  $V^{10} \subset H^{10}$ where the restriction of the Levi form is nondegenerate. 
Then obviously any such $V^{10}$ appears as the span
of the first $q$ vector fields in Catlin's boundary system,
and vice versa, every such span is a maximal Levi-nondegenerate subbundle.
%
%the span
%$$
%	V^{10}:= \C L_2 + \ldots + \C L_{q+1}
%$$
%as 
%%is any complex subbundle of maximal rank in a neighborhood of $M$,
%%to which the Levi form restriction is nondegenerate.


Next Catlin considers the Levi-orthogonal subbundle 
$$
	S^{10} := (V^{10})^\perp \subset H^{10}
$$ 
($T^{10}_{q+2}$ in Catlin's notation).
%as the Levi-orthogonal complement of $V^{10}$.
In particular, the subbundle {\em $S^{10}$ contains all Levi kernels $K^{10}_x$ at all points $x\in M$ near $p$},
even when $\dim K^{10}_x$ depends on $x$.
That makes the fiber $(V^{10}_x)^\perp$ unique
whenever the Levi rank at $x$ is the same as $p$,
even when $V^{10}_x$ itself may not be unique.
On the other hand, at points $x$ of higher Levi rank,
$(V^{10}_x)^\perp$ clearly depends on the choice of $V^{10}_x$.

The rest of Catlin's boundary system construction 
only depends on the subbundle $S^{10}$ 
rather than on $V^{10}$ and its chosen basis.



\subsection{Levi kernel inclusion of higher order for $S^{10}$}
As mentioned before, $S^{10}$ contains the Levi kernel at every point.
On the other hand, if $M$ is pseudoconvex,
we have shown in Lemma~\ref{1-ker-def} 
that $S^{10}$ is itself 
{\em contained in the Levi kernel up to order $1$ at $p$}
as defined in Definition~\ref{ker-1}.
That permits to use arbitrary sections of $S^{10}$
in the calculation of the quartic tensor $\t^4$:

\bc\Label{s10-tensor}
Let $M$ be a pseudoconvex hypersurface,
$V^{10}\subset H^{10}$
a maximal Levi-nondegenerate subbundle at $p\in M$,
and $S^{10}$ the Levi-orthogonal complement of $V^{10}$.
Then the quartic tensor $\t^4_p$ defined by Lemma~\ref{d2}
satisfies 
%\beq\Label{1-levi'}
$$
	\la \th_p, \t^4_p(L^4_p, L^3_p, L^2_p, L^1_p)\ra 
	= -i(L^4 L^3 \la \th, [L^2, L^1]\ra)_p
$$
for any $L^4, L^3 \in \C T$, $L^2, \1L^1\in S^{10}$ and $\th\in \Om^0$.
%\eeq
\ec

%\br
%Even if the first two arguments of $\t^4$ are arbitrary tangent directions of $M$,
%only the restriction  
%$$
%	\t^{40}_p \colon \C K_p \times \C K_p \times K^{10}_p \times \1{K^{10}_p}
%	\to \C Q_p
%$$
%is related to the rest of Catlin's boundary system construction.
%\er



%we obtain
%that any section of $S^{10}$ 
%is automatically 

%Then to have a boundary system in a neighborhood of $p$,
%it suffices to require that $V^{10}_p$ is a complement of the Levi kernel at $p$, i.e.\
%$$
%	H^{10}_p = V^{10}_p  \oplus K^{10}_p.
%$$

%
%Given any real hypersurface $M$,
%its {\em maximal Levi-nondegenerate subbundle}
%$$V^{10}\subset H^{10}$$ 
%through a point $p\in M$
%is any complex subbundle of maximal rank in a neighborhood of $M$,
%to which the Levi form restriction is nondegenerate.
%The maximality obviously implies that the bundle rank is equal the Levi form rank at $p$.
%
%For every maximal Levi-nondegenerate subbundle $V^{10}$,
%consider its Levi-orthogonal complement $(V^{10})^\perp$.
%Then $(V^{10})^\perp$ obviously contains the Levi kernel $K^{10}_x$ for every $x\in M$ near $p$.
%Thus, even when the $\dim K^{10}_x$ may depend on $x$



\subsection{Relation with the rest of Catlin's boundary system construction}
\Label{bs-2}
The remaining part of Catlin's construction is based on the higher order Levi form derivatives
\beq\Label{L-th}
	\6L \th := L^m \ldots L^3 \la \th, [L^2, L^1] \ra,
	\quad
	\6L = (L^m, \ldots, L^1),
\eeq
where $\th = \d r$ and $r$ is a defining function of $M$.
Then a boundary system 
\beq\Label{bd-sys}
	\6B = \{r_1, r_{q_2+2}, \ldots, r_\nu; L_2, \ldots, L_\nu  \},
	\quad q_2+2 \le \nu\le n,
\eeq
is constructed
together with associated weights
$$
	\a_1 =1 >  \a_2 =\ldots = \a_{q}=2 > \a_{q+1}\ge \ldots \ge \a_{\nu},
$$
where $r_1 = r$ is the given defining function,
$L_j$ and $r_j$ are respectively smooth $(1,0)$ vector fields
and smooth real functions in a neighborhood of $p$.
The construction proceeds by induction as follows.
Assuming a boundary system is constructed for given $\nu$,
define the next subbundle
$$
	T^{10}_{\nu+1} := \{ L\in T^{10}_{q_2+2} : \d r_{q_2+2}(L) =\ldots = \d r_{\nu}(L) =0 \}.
$$
Then count all previous $L_j$ and their conjugates with weight $\a_j$,
and consider a new vector field $L_{\nu+1}\in T^{10}_{\nu+1}$ and its conjugage,
whose weight $\a=\a_{\nu+1}$ is to be determined.
Now look for all lists 
$\6L = (L^m, \ldots, L^1)$ with each $L^k\in \{L_{q+2}, \ldots, L_{\nu+1}\}$,
which are of total weight $1$ and {\em ordered}, 
i.e.\ $L_j, \1L_j$ preceed $L_k, \1L_k$ whenever $j>k$,
such that 
\beq\Label{non-vanish}
	(\6L\d \rho)_p\ne 0.
\eeq
The list must contain the new vector field $L_{\nu+1}$ or its conjugate,
and the new weight $\a_{\nu+1}$ is chosen to be minimal possible
with that property.
Finally set either
$$
	r_{\nu+1}:= \Re L^{m-1}\ldots L^3 \la \th, [L^2, L^1] \ra
	\text{ or }
	r_{\nu+1}:= \Im L^{m-1}\ldots L^3 \la \th, [L^2, L^1] \ra	
$$
such that 
$$
	(L_{\nu+1} r_{\nu+1})_p\ne 0,
$$
which is always possible in view of \eqref{non-vanish}, 
since the first vector field in the list,
 $L^m$ is either $L_{\nu+1}$ or its conjugate.
Restating
Lemma~\ref{1-ker-def} and
Corollariy~\ref{s10-tensor}, 
we have:

\bc
Let $M$ be pseudoconvex hypersurface 
with Levi form of rank $q$ at $p$.
Fix a boundary system $\{L_2, \ldots, L_{q+1}\}$ at $p$.
Then $S^{10}= T^{10}_{q+2} = V^\perp$ for 
$V:=\span\{L_2, \ldots, L_{q+1}\}$.
Further, for any vector fields 
$L^4, L^3\in S^{10} + \1{S^{10}}$, $L^2\in S^{10}$, $L^1\in \1{S^{10}} $,
we have
$$
	L^3 \la \th, [L^2, L^1] \ra_p = 0,
$$
$$
	L^4 L^3 \la \th, [L^2, L^1] \ra_p = \t^{40}_p(L^4_p, L^3_p, L^2_p, L^1_p).
$$
In other words, for lists $\6L$ of length $3$, 
the derivative $(\6L\th)_p$ vanishes,
whereas for lists of length $4$, 
it only depends on the vector field values at $p$
and is given by the restricted quartic tensor $\t_4$
(regardless of the choice of the boundary system).
\ec

Thus via the quartic tensor restriction $\t^{40}_p$,
the nonvanishing condition in \eqref{non-vanish}
is reduced to a purely algebraic property
only depending on the vector fields' values at $p$.


%In particular, Example~\ref{diff-kernels}


%\bp
%Let $M$ be pseudoconvex hypersurface, and $p$ a point with Levi form of rank $q$.
%Then \eqref{bd-sys} is a boundary system if and only if
%for every $\nu+1$, there exists an ordered list 
%$\6L =(L^4, \ldots, L^1)$
%$$
%	\t_4(L^4_p, L^3_p, L^2_p, L^1_p)
%$$
%\ep




%\section{Multitype frames}
%

%\section{Containing stratifications}

%\subsection{Quartic tensor cokernels}
%
%\bd
%Let $V_1, \ldots, V_m$ be complex vector spaces
%and 
%$$
%	\t\colon V_1\times \ldots \times V_m \to \C
%$$
%a complex-multilinear map.
%We define the {\em cokernel of $\t$ in $V_1$} 
%to be the span in the dual $V_1^*$ of all functionals
%$$
%	\psi(v) = \t(v, v_2, \ldots, v_m),
%	\quad
%	(v_2,\ldots, v_m) \in V_2\times\ldots \times V_m.
%$$
%More generally, define the 
%{\em cokernel of a multilinear map}
%$$
%	\t\colon V_1\times \ldots \times V_m \to W
%$$
%is that of the associated map
%$$
%	\t\colon V_1\times \ldots \times V_m \times W^*\to \C.
%$$
%\ed
%
%
%\bp
%Let $M$ be a pseudoconvex hypersurface
%with quartic tensor $\t^4_p$ at $p\in M$,
%and let
% $\6E_p\subset \C K_p$ be the dual subspace 
%generated by $\t^{40}$.
%\ep
%
%
%\be
%Let $M$ be the tube quartic
%$$
%	\rho:= - 2\Re w + (z+\z)^4 =0.
%$$
%Then 
%$$
%	\t^{40}_0 = 4! (dz+d\z)\otimes (dz + d\z) \otimes dz \otimes d\z
%$$
%\ee



\begin{thebibliography}{BER96}

\bibitem[BS16]{BS16} 
S. Biard, E. J. Straube.
L2-Sobolev theory for the complex Green operator. 
Notes of four lectures given by the second author at the January 2016 School in Complex Analysis and Geometry at the Tsinghua Sanya International Mathematics Forum, Sanya, China;
{\sf https://arxiv.org/abs/1606.00728}

\bibitem[C83]{C83}  David Catlin. Necessary conditions for subellipticity of the $\bar\d$-Neumann problem. {\em Ann. of Math.} (2) 117 (1983), no. 1, 147--171.

\bibitem[C84a]{C84a}  David Catlin. Global regularity of the $\bar\d$-Neumann problem. Complex analysis of several variables (Madison, Wis., 1982), 39--49, Proc. Sympos. Pure Math., 41, Amer. Math. Soc., Providence, RI, 1984.

\bibitem[C84b]{C84b}  David Catlin. Boundary invariants of pseudoconvex domains. {\em Ann. of Math. (2)}, {\bf 120}(3):  529--586, 1984.

\bibitem[C87]{C87} David Catlin. Subelliptic estimates for the $\bar\d$-Neumann problem on pseudoconvex domains. {\em Ann. of Math. (2)}, {\bf 126} (1): 131--191, 1987.

\bibitem[D80]{D80}  John P. D'Angelo.
Subelliptic estimates and failure of semicontinuity for orders of contact. 
{\em Duke Math. J.} 47 (1980), no. 4, 955--957.

\bibitem[D82]{D82} John P. D'Angelo. Real hypersurfaces, orders of contact, and applications. 
{\em Ann. of Math. (2)}, {\bf 115} (3): 615--637, 1982.

\bibitem[E]{E-jdg} Ebenfelt. New invariant tensors in CR structures and a normal form for real hypersurfaces at a generic Levi degeneracy. {\em J. Differential Geom.} 50 (1998), no. 2, 207--247. 

\bibitem[K64a]{K64a} J. J. Kohn. Harmonic integrals on strongly pseudo-convex manifolds. I. {\em Ann. of Math.} (2) 78 1963, 112--148.

\bibitem[K64b]{K64b} J. J. Kohn. Harmonic integrals on strongly pseudo-convex manifolds. II. {\em Ann. of Math.} (2) 79 1964, 450--472.

\bibitem[K79]{K79} J. J. Kohn. Subellipticity of the $\bar\d$-Neumann problem on pseudo-convex domains: sufficient conditions. {\em Acta Math.} 142 (1979), no. 1-2, 79--122.

%\bibitem[M02]{M02}
%J. McNeal. A Sufficient Condition for Compactness of the 
%$\bar\d$-Neumann Operator.
%Journal of Functional Analysis 195, 190--205 (2002).

\bibitem[W95]{W} S. M. Webster. The holomorphic contact geometry of a real hypersurface, Modern Methods in Complex Analysis (T. Bloom et al, eds.), Annals of Mathematics Studies 137, Princeton University Press, Princeton, N.J., 1995, pp. 327--342.

\end{thebibliography}

\end{document}




